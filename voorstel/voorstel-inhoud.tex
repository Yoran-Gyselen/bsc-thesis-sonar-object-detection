%---------- Inleiding ---------------------------------------------------------

\section{Inleiding}%
\label{sec:inleiding}

Exail Robotics is op zoek naar een oplossing om bepaalde objecten te detecteren in grote datasets gegenereerd door metingen met behulp van sonar. Binnen deze probleemstelling zijn er heel wat mogelijkheden tot het uitdiepen en onderzoeken van bepaalde details. Aangezien het officiële onderwerp nog niet gekend is (cf. supra), zal ik hieronder proberen enkele mogelijke onderwerpen duidelijker voor te stellen.

\subsection{Optimalisatie van Objectdetectie op Sonarbeelden}

\textbf{Hoofdvraag:} Hoe kunnen Deep Learning-modellen worden geoptimaliseerd om objecten nauwkeurig te detecteren en classificeren op sonar-afbeeldingen? \\

De bedoeling binnen dit onderzoek is om verschillende technieken te vergelijken om Deep Learning-modellen te optimaliseren en verbeteren. Specifiek zou dit gaan om het onderzoeken van verschillende neurale netwerkarchitecturen (bv. CNNs, YOLO, of transformer-based modellen). Daarnaast wil ik preprocessing-technieken zoals ruisonderdrukking of contrastverhoging onderzoeken om het model te verbeteren. Ook zou ik graag de mogelijkheid tot transfer learning-technieken onderzoeken om training te verbeteren. \\

\textbf{Uitkomst:} Een benchmark van verschillende technieken en een prototype van een optimaal model.

\subsection{Labeling van Sonardata met behulp van Semi- of Self-supervised Learning}

\textbf{Hoofdvraag:} Hoe kan het gebruik van semi- of self-supervised learning het labelproces versnellen met minimaal verlies in nauwkeurigheid? \\

De bedoeling binnen dit onderzoek is om een self-supervised strategie te ontwikkelen met behulp van verschillende technieken (bv. BYOL). Ook wil ik een analyse doen van hoeveel gelabelde data nodig is voor een goed presterend model. \\

\textbf{Uitkomst:} Een proces om detectie te verbeteren met minimale gelabelde data.

\subsection{Anomaliedetectie voor Automatische Identificatie van Onbekende Objecten}

\textbf{Hoofdvraag:} Hoe kunnen afwijkende objecten automatisch worden geïdentificeerd op sonar-beelden, zonder dat er specifieke labels voor deze objecten zijn? \\

De bedoeling binnen dit onderzoek is om verschillende unsupervised learning technieken (zoals autoencoders of clustering-methoden) toe te passen om abnormale patronen te detecteren en deze te vergelijken met elkaar. Daarnaast wil ik deze technieken evalueren op sonar-specifieke datasets. \\

\textbf{Uitkomst:} Een framework voor het detecteren van onbekende objecten.

\subsection{Vergelijking van Data-Augmentatie technieken voor Sonardata}

\textbf{Hoofdvraag:} Welke data-augmentatiestrategieën verbeteren de prestaties van detectiemodellen op sonar-beelden? \\

De bedoeling binnen dit onderzoek is om verschillende technieken om extra data te genereren uit een bestaande dataset. Dit kan gaan van simpele technieken zoals rotatie, reflectie, of intensiteitsveranderingen tot het trainen van bepaalde modellen om bijkomende (nieuwe) data te genereren. \\

\textbf{Uitkomst:} Richtlijnen en best practices voor data-augmentatie in sonar-projecten.

\subsection{Explainable AI (XAI) voor Objectdetectie op Sonardata}

\textbf{Hoofdvraag:} Hoe kunnen de beslissingen van objectdetectiemodellen op sonarbeelden beter geïnterpreteerd en verklaard worden? \\

De bedoeling binnen dit onderzoek is om verschillende XAI-technieken toe te passen zoals SHAP, LIME of saliency maps. Ook zou ik graag een visueel interpretatiemodel ontwikkelen specifiek voor sonar-data. Tot slot wil ik een enquête doen naar de acceptatie en bruikbaarheid van XAI door eindgebruikers. \\

\textbf{Uitkomst:} Tools en inzichten die de transparantie van AI-modellen verbeteren.

%---------- Stand van zaken ---------------------------------------------------

\section{Literatuurstudie}%
\label{sec:literatuurstudie}

Objectdetectie is een tak binnen het domein van computer vision dat gericht is op het identificeren en lokaliseren van objecten binnen beelddata (zoals foto's en video's). Objectdetectie wordt gebruikt in verschillende domeinen, zoals beveiligingssystemen (bv. om inbrekers te detecteren) of de medische wereld (bv. om tumoren op te sporen). Door de jaren heen is objectdetectie aanzienlijk geëvolueerd dankzij de vooruitgang in Deep Learning en de grote beschikbaarheid van datasets met beeldmateriaal. \autocite{He_2016}

Objectdetectie combineert twee belangrijke taken in computer vision: objectlokalisatie en objectclassificatie. Objectlokalisatie bepaalt de positie van objecten, meestal in de vorm van bounding boxes \autocite{Tompson_2015}, terwijl objectclassificatie bepaalt tot welke categorie een gedetecteerd object behoort. Samen geeft dit de mogelijkheid tot het herkennen van verschillende objecten op één foto.

Sonardata vormt een unieke uitdaging voor objectdetectie vanwege de specifieke kenmerken van akoestische beelden, zoals ruis, lage resolutie en reflecties. De meeste technieken zijn echter oorspronkelijk ontwikkeld voor visuele data. Er zijn dus verschillende aanpassingen nodig om deze methoden effectief te laten werken op sonarafbeeldingen. Één van deze aanpassingen zit hem in het preprocessen van de data met behulp van filters om ruis te onderdrukken en het contrast te verbeteren. Ook kan er gebruik gemaakt worden van aangepaste modellen die speciaal kunnen worden ontworpen om beter te werken met sonardata. Dit kan door rekening te houden met hoe objecten in de data ruimtelijk met elkaar verbonden zijn en met specifieke geluidspatronen die in sonarbeelden voorkomen. \autocite{Karimanzira_2020}

Daarnaast is het formaat van de data cruciaal. Er wordt (hoogstwaarschijnlijk) gebruik gemaakt van stacked GeoTIFF. Hierbij wordt het GeoTIFF formaat gebruikt om meerdere lagen geografische data te combineren in één bestand. GeoTIFF is een uitbreiding van het standaard TIFF (Tagged Image File Format) en wordt gebruikt om rasterafbeeldingen te koppelen aan geografische coördinaten. \autocite{Ritter_1997}

% Voor literatuurverwijzingen zijn er twee belangrijke commando's:
% \autocite{KEY} => (Auteur, jaartal) Gebruik dit als de naam van de auteur
%   geen onderdeel is van de zin.
% \textcite{KEY} => Auteur (jaartal)  Gebruik dit als de auteursnaam wel een
%   functie heeft in de zin (bv. ``Uit onderzoek door Doll & Hill (1954) bleek
%   ...'')

%---------- Methodologie ------------------------------------------------------
\section{Methodologie}%
\label{sec:methodologie}

Afhankelijk van welk onderwerp gekozen wordt, zal de methodologie ietwat anders zijn. In grote lijnen komen de stappen in het onderzoek wel overeen. Eerst en vooral wordt de data verzameld. Deze data wordt aangeleverd in stacked GeoTIFF-formaat. Daarna worden bepaalde preprocessingtechnieken toegepast op de data, zoals ruisonderdrukking, normalisatie en het extraheren van relevante lagen uit de GeoTIFF-bestanden, zoals intensiteit en topografische gegevens. Vervolgens wordt indien nodig data-augmentatie toegepast, bijvoorbeeld door rotatie of schaling, om het model robuuster te maken. Het model wordt ontwikkeld. Dit is enorm afhankelijk van het onderwerp en zal later verder uitgediept worden. Uiteindelijk wordt het model geëvalueerd met behulp van verschillende metrieken. Afhankelijk van deze metrieken wordt het model geoptimaliseerd om de nauwkeurigheid te verbeteren en de fout te verkleinen. Ten slotte wordt het model getest op een onafhankelijke dataset en geëvalueerd op prestaties in real-world scenario's, zoals verschillende wateromstandigheden en objecttypes.

%---------- Verwachte resultaten ----------------------------------------------
\section{Verwacht resultaat, conclusie}%
\label{sec:verwachte_resultaten}

Ik verwacht, onafhankelijk van het onderwerp, een werkend model te ontwikkelen dat het specifiek beschreven probleem in het onderwerp kan oplossen. Dit doe zal ik doen door geavanceerde ML- en Deep Learning-technieken te combineren met gespecialiseerde preprocessingmethoden voor sonardata. Het uiteindelijke model zal niet alleen een waardevol hulpmiddel zijn voor mijn stagebedrijf, maar kan ook als basis dienen voor verdere toepassingen in maritiem onderzoek en sonarbeeldanalyse.

