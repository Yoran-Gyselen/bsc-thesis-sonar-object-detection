%%=============================================================================
%% Inleiding
%%=============================================================================

\chapter{Inleiding}%
\label{ch:inleiding}

Objectdetectie heeft de afgelopen jaren enorme vooruitgang geboekt dankzij de opkomst van deep learning en de beschikbaarheid van grote, gelabelde datasets. In domeinen zoals computervisie, waar overvloedige trainingsdata gemakkelijk toegankelijk is, hebben deze technieken indrukwekkende prestaties bereikt. Echter beschikken niet alle vakgebieden over dergelijke datasets. In gespecialiseerde domeinen, zoals sonarbeeldanalyse, is gelabelde data schaars, wat het trainen van nauwkeurige detectiemodellen bemoeilijkt. Dit onderzoek richt zich op het verkennen van alternatieve leermethoden die deze afhankelijkheid van handmatige annotatie kunnen verminderen, zonder in te boeten op de prestaties van het model.

\section{Probleemstelling}%
\label{sec:probleemstelling}

Om een model te trainen dat met hoge precisie objecten in afbeeldingen kan herkennen en aanduiden, is een grote hoeveelheid gelabelde data nodig. Dit betekent dat naast de afbeeldingen zelf ook informatie over de positie van het object op de afbeelding beschikbaar moet zijn. Na de training is het de bedoeling dat het model deze informatie kan voorspellen op ongekende afbeeldingen. Sinds de opkomst van objectdetectie binnen het veld van machine learning zijn verschillende datasets openbaar beschikbaar gesteld die vrij gebruikt mogen worden om een dergelijk model te trainen. Dit geldt echter niet voor sonardata, om meerdere redenen. \\

Allereerst is objectdetectie op sonarbeelden een nicheprobleem. Hierdoor beschikken slechts weinig mensen over de kennis en expertise om zo’n dataset samen te stellen en, nog belangrijker, correct te annoteren. Daarnaast is er het -- misschien nog grotere -- probleem van de data zelf. Dit type gegevens kan niet eenvoudig met een camera worden verzameld; er is een gespecialiseerde sonarinstallatie voor nodig. Voor bedrijven is de aanschaf van zo’n installatie vaak weinig rendabel. Wel worden dergelijke systemen gebruikt voor militaire doeleinden, maar de data die hieruit voortkomt, is om veiligheidsredenen vrijwel altijd geclassificeerd. \\

De probleemstelling is dus tweeledig: er is weinig gelabelde sonardata voor objectdetectie beschikbaar, en het annoteren van een dergelijke dataset is moeilijk, tijdrovend en kostbaar.

\section{Onderzoeksvraag}%
\label{sec:onderzoeksvraag}

Om kosten en tijd te besparen, zou het dus ideaal zijn als er zo min mogelijk annotatie van de dataset nodig is. Bij supervised learning is dit echter nagenoeg onmogelijk, aangezien het model juist getraind wordt op basis van het verband tussen de afbeelding en de annotatie. Er bestaan echter veelbelovende alternatieven, zoals \gls{ssl} en \gls{self-sl}, om dit probleem te overbruggen. Deze technieken maken gebruik van ongesuperviseerde data om hun performantie te verbeteren en beperken zo de afhankelijkheid van gelabelde data. Moderne \gls{ssl} en \gls{self-sl} methoden hebben indrukwekkende resultaten laten zien in domeinen zoals computer vision, maar hun toepassing op domeinspecifieke datasets, zoals sonar, is nog relatief onbekend terrein. \\

De hoofdvraag van dit onderzoek is daarom: Op welke manieren kan het gebruik van \gls{ssl} of \gls{self-sl} het labelproces versnellen zonder een significant verlies in nauwkeurigheid?

\section{Onderzoeksdoelstelling}%
\label{sec:onderzoeksdoelstelling}

Het onderzoek verwacht aan te tonen dat \gls{ssl} en \gls{self-sl} effectief kunnen worden ingezet om het labelproces bij sonarobjectdetectie aanzienlijk te versnellen. Door technieken zoals \gls{simclr}, \gls{byol}, Pseudo-Labeling en FixMatch te gebruiken, wordt verwacht dat de performantie van het model sterk verbeterd door ongesuperviseerde sonardata, waardoor de behoefte aan grootschalige gelabelde datasets afneemt. Daarnaast zal een analyse inzicht geven in de minimale hoeveelheid gelabelde data die nodig is om vergelijkbare of betere prestaties te behalen dan met volledig supervised-learning methoden. \\

Dit resulteert in een efficiëntere en kosteneffectieve aanpak voor objectdetectie in sonarbeelden, zonder verlies van nauwkeurigheid, en biedt een waardevolle methodologie voor verdere toepassingen in domeinen waar gelabelde data schaars is.

\section{Beschikbaarheid van de code}

Alle code die gebruikt werd voor de experimenten en implementatie in deze bachelorproef is publiek beschikbaar via een GitHub-repository. De repository bevat de volledige implementatie van de gebruikte modellen, inclusief de baseline, het \gls{ssl}-model en het \gls{self-sl}-model. Daarnaast zijn ook de scripts voor training, evaluatie en visualisatie van de resultaten opgenomen. De code is modulair opgezet en voorzien van duidelijke documentatie, zodat reproduceerbaarheid en verdere uitbreidingen mogelijk zijn. De repository is te vinden op: \href{https://github.com/YoranGyselen-Hogent/bap-2425-yorangyselen}{github.com/YoranGyselen-Hogent/bap-2425-yorangyselen}. Neem voor vragen over dit onderzoek contact op met de auteur via het e-mailadres \href{mailto:yoran@gyselen.be}{yoran@gyselen.be}.

\section{Opzet van deze bachelorproef}%
\label{sec:opzet-bachelorproef}

De rest van deze bachelorproef is als volgt opgebouwd: \\

In Hoofdstuk~\ref{ch:stand-van-zaken} wordt een overzicht gegeven van de stand van zaken binnen het onderzoeksdomein, op basis van een literatuurstudie. \\

In Hoofdstuk~\ref{ch:methodologie} wordt de methodologie toegelicht en worden de gebruikte onderzoekstechnieken besproken om een antwoord te kunnen formuleren op de onderzoeksvragen. \\

In Hoofdstuk~\ref{ch:experimenten} wordt het eigenlijke onderzoek uitgevoerd. In deze fase wordt het experimentele kader verduidelijk en zal er dieper ingegaan worden op de effectieve uitvoering van de verschillende experimenten. \\

In Hoofdstuk~\ref{ch:resultaten-evaluatie} worden de getrainde modellen geëvalueerd op basis van verschillende criteria. Er wordt een vergelijking gemaakt tussen -- onder andere -- de performantie van elk model en er wordt bepaald welk model de beste is. Daarnaast wordt de praktische toepassing van de verschillende modellen geëvalueerd. Ook zullen enkele experts in sonaranalyse de bruikbaarheid van de resultaten beoordelen en aanbevelingen geven voor verdere verbeteringen. \\

In Hoofdstuk~\ref{ch:conclusie}, tenslotte, wordt de conclusie gegeven en een antwoord geformuleerd op de onderzoeksvragen. Daarbij wordt ook een aanzet gegeven voor toekomstig onderzoek binnen dit domein.