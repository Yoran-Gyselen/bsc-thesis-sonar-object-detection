%%=============================================================================
%% Samenvatting
%%=============================================================================

% TODO: De "abstract" of samenvatting is een kernachtige (~ 1 blz. voor een
% thesis) synthese van het document.
%
% Een goede abstract biedt een kernachtig antwoord op volgende vragen:
%
% 1. Waarover gaat de bachelorproef?
% 2. Waarom heb je er over geschreven?
% 3. Hoe heb je het onderzoek uitgevoerd?
% 4. Wat waren de resultaten? Wat blijkt uit je onderzoek?
% 5. Wat betekenen je resultaten? Wat is de relevantie voor het werkveld?
%
% Daarom bestaat een abstract uit volgende componenten:
%
% - inleiding + kaderen thema
% - probleemstelling
% - (centrale) onderzoeksvraag
% - onderzoeksdoelstelling
% - methodologie
% - resultaten (beperk tot de belangrijkste, relevant voor de onderzoeksvraag)
% - conclusies, aanbevelingen, beperkingen
%
% LET OP! Een samenvatting is GEEN voorwoord!

%%---------- Samenvatting -----------------------------------------------------
% De samenvatting in de hoofdtaal van het document

\chapter*{Samenvatting}

Sinds de opkomst en popularisatie van AI-modellen is data steeds een cruciale resource geweest. Voor simpele modellen is de benodigde data vaak ook simpel van vorm en is er (relatief) weinig van nodig om een performant en goed werkend model te creëren. Echter stijgen de data-requirements voor grotere en complexere modellen exponentieel. De benodigde data om een objectdetectiemodel voor sonardata te trainen zorgt voor moeilijkheden: dit soort datasets zijn online niet off-the-shelf beschikbaar en zijn dus zeer tijdrovend en kostbaar om te maken. De hoofdvraag van dit onderzoek is daarom: Op welke manieren kan het gebruik van semi- of self-supervised learning het labelproces versnellen zonder een significant verlies in nauwkeurigheid? Door verschillende technieken toe te passen, zal een pretraining-strategie ontwikkeld worden die gebruikmaakt van ongelabelde data om representaties aan te leren. Vervolgens zal onderzocht worden hoeveel gelabelde data nodig is om een goed presterend detectiemodel te trainen. Het doel is een methodologie te ontwikkelen die de afhankelijkheid van handmatig gelabelde data minimaliseert, terwijl de prestaties van het detectiemodel behouden blijven. De resultaten kunnen bijdragen aan efficiëntere workflows voor data-analyse in sonarbeeldvorming en andere domeinspecifieke contexten.
