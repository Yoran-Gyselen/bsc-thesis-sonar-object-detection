%%=============================================================================
%% Voorwoord
%%=============================================================================

\chapter*{\IfLanguageName{dutch}{Woord vooraf}{Preface}}%
\label{ch:voorwoord}

%% TODO:
%% Het voorwoord is het enige deel van de bachelorproef waar je vanuit je
%% eigen standpunt (``ik-vorm'') mag schrijven. Je kan hier bv. motiveren
%% waarom jij het onderwerp wil bespreken.
%% Vergeet ook niet te bedanken wie je geholpen/gesteund/... heeft

Voor u ligt mijn bachelorproef over het gebruik van semi- en self-supervised learningtechnieken voor objectdetectie op sonarbeelden. Dit onderzoek richt zich op de vraag of dergelijke technieken het labelproces kunnen versnellen zonder significant verlies in nauwkeurigheid. Het doel is een efficiëntere methode ontwikkelen voor het verwerken van sonardata. \\

Mijn interesse in dit onderwerp ontstond vanuit een combinatie van mijn passie voor machine learning en de uitdagingen die ik tijdens mijn stage tegenkwam. Het verwerken en labelen van sonardata bleek een tijdrovend en arbeidsintensief proces te zijn. Dit gaf me het idee om innovatieve methoden te verkennen om dit te optimaliseren. Semi- en self-supervised learning boden een veelbelovende oplossing, en ik was benieuwd of deze technieken in de praktijk daadwerkelijk een verschil konden maken. \\

Het schrijven van deze bachelorproef was een enorm leerzaam, maar uitdagend proces, waarin ik veel heb bijgeleerd over machine learning en de praktische toepassingen ervan binnen de industrie. Dit onderzoek zou niet mogelijk zijn geweest zonder de steun en begeleiding van verschillende mensen, aan wie ik graag mijn dank wil uitspreken. \\

Allereerst wil ik mijn promotor, mevr. Chantal Teerlinck, bedanken voor de goede begeleiding, feedback en inzichten tijdens dit traject. Haar ervaring en ondersteuning hebben me geholpen om dit onderzoek in de juiste richting te sturen. Daarnaast wil ik mijn co-promotor, mevr. Stefanie Duyck, bedanken voor haar betrokkenheid, kennis en praktische inzichten vanuit de bedrijfswereld, wat een belangrijke meerwaarde vormde voor dit onderzoek. \\

Ook wil ik mijn dank uitspreken aan Exail Robotics Belgium voor de kans om mijn bachelorproef binnen hun organisatie uit te voeren. De toegang tot relevante -- en zeer waardevolle -- data en de begeleiding vanuit het team hebben een cruciale rol gespeeld in het realiseren van dit onderzoek. Tot slot wil ik mijn familie en vrienden bedanken voor hun steun en aanmoediging gedurende mijn studietraject. \\

Ik hoop dat deze bachelorproef een bijdrage kan leveren binnen het domein van objectdetectie op sonarbeelden en machine learning in het algemeen.