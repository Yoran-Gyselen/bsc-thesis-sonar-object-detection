\section{Objectdetectie in sonarafbeeldingen}

\subsection{Definitie en gebruik op sonarafbeeldingen}

Objectdetectie is een tak binnen het domein van computer vision dat gericht is op het identificeren en lokaliseren van objecten binnen beelddata (zoals foto's en video's). Dit wordt gebruikt in verschillende domeinen, zoals beveiligingssystemen (bv. om inbrekers te detecteren) of de medische wereld (bv. om tumoren op te sporen). Door de jaren heen is objectdetectie aanzienlijk geëvolueerd dankzij de vooruitgang in Deep Learning en de grote beschikbaarheid van datasets met beeldmateriaal. \autocite{He_2016} \\

Objectdetectie combineert twee belangrijke zaken in computer vision: objectlokalisatie en objectclassificatie. Objectlokalisatie bepaalt de positie van objecten, meestal in de vorm van \glspl{bounding_box} \autocite{Tompson_2015}, terwijl objectclassificatie bepaalt tot welke categorie een gedetecteerd object behoort. Samen geeft dit de mogelijkheid tot het herkennen van verschillende soorten objecten op één foto. \\

Objectdetectie heeft vele toepassingen, ook in domeinen die misschien niet zo voor de hand liggend zijn. Één van deze specialisaties binnen de -- algemene -- objectdetectie is objectdetectie op sonardata. Dit domein is over de laatste jaren erg gegroeid \autocite{Wang_2024}

Traditioneel worden gesuperviseerde methoden -- zoals Faster \gls{rcnn}, \gls{yolo} en \gls{ssd} -- gebruikt voor objectdetectie. \autocite{Redmon_2016} Deze modellen presteren uitstekend bij voldoende gelabelde data, maar de annotatiekosten en tijdsinvestering vormen een grote belemmering, vooral bij complexe datasets zoals sonar. Sonardata vereist namelijk gespecialiseerde kennis voor het labelen, wat de annotatie nog uitdagender maakt. \autocite{Long_2015} \\

\subsection{Typische uitdagingen bij sonarobjectdetectie}

\lipsum[1-3]

\subsection{Overzicht van bestaande technieken}

\lipsum[1-3]

\subsection{Specifieke toepassingen}

\lipsum[1-3]