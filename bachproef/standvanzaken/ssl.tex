\section{Semi-supervised learning: principes en technieken}

\subsection{Definitie en waarde binnen objectdetectie}

Binnen het domein van machine learning bestaan er verschillende technieken om een model te trainen. Meestal wordt er gesproken van twee grote stromingen: supervised learning en unsupervised learning. Bij supervised learning wordt er gebruik gemaakt van een dataset en een uitkomst (hetgeen het model uiteindelijk moet kunnen voorspellen). Dit kan een label zijn of een bepaalde numerieke waarde. Belangrijk is dat zowel de input als de gewenste output gegeven zijn. Het model leert dus het verband tussen de twee. Bij unsupervised learning zijn er geen verwachte outputs. De volledige dataset wordt door het model gebruikt om patronen in te herkennen. Unsupervised learning wordt daarom ook meestal gebruikt om verkennende data-analyse uit te voeren. Echter hebben beide methoden enkele nadelen. Bij supervised learning is het traag en duur om alle data op een correcte manier te labelen. Unsupervised learning heeft dit probleem niet, maar heeft een beperkt aantal toepassingen en is minder accuraat. Een alternatief is semi-supervised learning, wat een compromis tussen zowel supervised als unsupervised learning is. \autocite{C_A_Padmanabha_Reddy_2018} \\

\subsection{Veelgebruikte SSL-methoden}

\subsubsection{Pseudo-labeling}

\subsubsection{Consistency Regularization}

\subsubsection{MixMatch en FixMatch}