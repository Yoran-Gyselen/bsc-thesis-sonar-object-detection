\paragraph{Inleiding}

Objectdetectie in domeinspecifieke contexten zoals sonarbeeldvorming wordt vaak gehinderd door een gebrek aan gelabelde data. Traditioneel vereisen gesuperviseerde modellen grote hoeveelheden handmatig gelabelde gegevens om effectieve detectie en classificatie te leren. Semi-supervised en self-supervised learning bieden echter veelbelovende alternatieven door gebruik te maken van grote hoeveelheden ongesuperviseerde data om representaties te leren. Deze literatuurstudie bespreekt de huidige technieken en hun toepassing, met een specifieke focus op de unieke uitdagingen van sonardata.