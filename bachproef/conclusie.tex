%%=============================================================================
%% Conclusie
%%=============================================================================

\chapter{Conclusie}%
\label{ch:conclusie}

\section{Samenvatting van bevindingen}

Op basis van de uitgevoerde experimenten kan geconcludeerd worden dat zowel \gls{ssl} als \gls{self-sl} effectieve strategieën zijn om de afhankelijkheid van gelabelde data bij objectdetectie in sonarbeelden te reduceren. Het volledig gesuperviseerde baseline-model, Faster \gls{rcnn}, behaalde een hoge nauwkeurigheid wanneer het getraind werd op de volledige gelabelde dataset (100\%), met een \gls{map} van 0.7717. Bij afname van het gelabelde aandeel daalden de prestaties echter significant, met \gls{map}-scores van respectievelijk 0.7847 (50\%)\footnote{De hogere performantie van het 50\%-model ten opzichte van het 100\%-model kan verklaard worden door \gls{overfitting}: het model dat op 100\% van de data is getraind, heeft mogelijk meer ruis of irrelevante patronen in de volledige dataset geleerd. Bij 50\% gelabelde data is de kans groter dat het model generaliseert naar meer robuuste kenmerken, wat resulteert in een iets hogere \gls{map}.}, 0.6152 (10\%), 0.5298 (5\%) en 0.2799 (1\%). Deze trend onderstreept de kwetsbaarheid van conventionele supervised modellen in situaties met beperkte annotaties. \\

De toepassing van \gls{ssl} met FixMatch bood in dit opzicht een duidelijke verbetering. Wanneer slechts 5\% of 10\% van de data gelabeld was, behaalde FixMatch \gls{map}-scores van 0.6649 en 0.6828. Deze scores liggen beduidend hoger dan die van het supervised baseline-model bij dezelfde splits, wat aantoont dat het model in staat is om op effectieve wijze gebruik te maken van de resterende ongelabelde data via pseudo-labeling en consistency regularisatie. Hoewel FixMatch niet de absolute topresultaten behaalde, vormt het een krachtige techniek voor situaties waarin annotaties beperkt beschikbaar zijn, zonder dat pretraining noodzakelijk is. \\

De best presterende alternatieve aanpak bleek echter de \gls{self-sl} strategie met \gls{byol}. Door een \gls{byol}-model te pretrainen op ongelabelde sonarbeelden en deze gewichten vervolgens te gebruiken als backbone voor een Faster \gls{rcnn} model, konden bij 5\% en 10\% gelabelde data \gls{map}-scores van respectievelijk 0.6452 en 0.7230 bereikt worden. Met name de score van 0.7230 bij 10\% gelabelde data benadert de prestaties van het volledig gesuperviseerde model, ondanks dat het met slechts een tiende van de annotaties getraind werd. Dit suggereert dat de representaties geleerd tijdens de \gls{byol}-pretraining bijzonder effectief generaliseren naar de downstream detectietaak. \\

Samenvattend tonen de resultaten aan dat zowel FixMatch als \gls{byol} waardevolle technieken zijn in data-schaarsere omgevingen. FixMatch biedt een efficiënte manier om ongelabelde data direct tijdens training te benutten, terwijl \gls{byol} robuuste representaties leert die downstream training aanzienlijk versterken. De \gls{self-sl} aanpak met \gls{byol} leverde de beste algehele prestaties in low-label settings, en vormt daarmee een veelbelovende richting voor verdere optimalisatie van objectdetectie in gespecialiseerde domeinen zoals sonarbeeldanalyse.

\section{Reflectie op de methodologie}

Bij reflectie op de gehanteerde methodologie kan worden vastgesteld dat de keuzes die aan de basis van dit onderzoek lagen over het algemeen sterk onderbouwd en passend waren binnen de onderzoeksvraag. De geselecteerde dataset bleek bijzonder geschikt: ze bood voldoende variatie en detail om de relevantie van het probleem van objectdetectie in sonarbeelden realistisch te benaderen. Bovendien sloot de datastructuur goed aan bij de gekozen leerstrategieën, wat toeliet om zowel supervisie als pretraining correct te benutten. \\

Ook de selectie van modellen -- Faster \gls{rcnn}, FixMatch en \gls{byol} -- bleek achteraf gezien goed doordacht. Deze modellen zijn zowel conceptueel uitdagend als voldoende krachtig om degelijke experimentele resultaten op te leveren. Ze boden een goed evenwicht tussen theoretische diepgang en praktische toepasbaarheid. Tegelijkertijd moet worden erkend dat de implementatie van vooral de \gls{ssl} en \gls{self-sl} modellen niet zonder moeilijkheden verliep. Het FixMatch-model, hoewel goed gedocumenteerd in literatuur, bleek in praktijk lastig op punt te stellen: het verkrijgen van stabiele en competitieve prestaties vereiste veel iteraties, hyperparameter-tuning en finetuning van augmentatiestrategieën. Het model bleek gevoelig voor kleine variaties in input en training, wat het reproduceerbaar werken bemoeilijkte. \\

De moeilijkheden bij \gls{byol} waren nog fundamenteler in de vroege fase. Initieel bleek dat een fout in de modelconfiguratie ertoe leidde dat slechts een beperkt deel van de ResNet-architectuur effectief getraind werd. Deze fout beperkte de prestaties drastisch tot \gls{map}-scores onder 0.15, waardoor diepgaand debuggingwerk noodzakelijk was. Pas na het herstellen van deze fout kon het \gls{byol}-model zijn potentieel laten zien. Daarnaast was het uitvoeren van optimalisaties niet evident: kleine aanpassingen aan architectuur of trainingsstrategie hadden vaak geen effect of zelfs negatieve impact. Dit toont aan dat dergelijke modellen fragiel kunnen zijn en dat hun succes sterk afhankelijk is van zorgvuldige afstemming van alle onderdelen van de trainingspipeline. \\

Verder speelde de computationele requirements een belangrijke rol. Door de (relatief) lange trainingstijden en de nood aan krachtige \gls{gpu}-resources, moest het onderzoek in iteraties worden uitgevoerd en werd experimenteel werk vaak vertraagd. Sommige experimenten konden niet in parallel worden opgezet, wat tijdsdruk met zich meebracht. Deze beperkingen illustreren de praktische uitdagingen bij het toepassen van geavanceerde leermethodes in een onderzoeksomgeving die beperkt wordt qua resources. \\

Ondanks deze obstakels was de gekozen methodologie waardevol. De moeilijkheden brachten leerzame inzichten met zich mee over de robuustheid en gevoeligheid van moderne machine learning-technieken in domeinspecifieke contexten. Het experiment gaf een realistisch beeld van wat er komt kijken bij het inzetten van geavanceerde \gls{ssl} en \gls{self-sl} modellen in een complexe toepassing zoals sonarbeeldanalyse.

\section{Voorstellen voor verder onderzoek}

Hoewel dit onderzoek aantoont dat zowel \gls{ssl} als \gls{self-sl} grote voordelen kunnen bieden bij objectdetectie op sonarbeelden, blijven er nog verschillende onderzoekspistes open die verdere verkenning verdienen. Een eerste richting is het vergelijken van alternatieve \gls{self-sl} pretrainingstrategieën. Hoewel \gls{byol} in dit onderzoek werd gekozen vanwege zijn stabiliteit en prestaties zonder negatieve paren, zijn er andere competitieve methoden zoals \gls{simclr}, \gls{moco} en \gls{dino}. Een vergelijking van deze methodes zou kunnen uitwijzen welke het meest geschikt is voor de specifieke kenmerken van sonarbeeldvorming, zoals lage resolutie, ruis en beperkte kleurinformatie. Daarnaast kan onderzocht worden hoe verschillende backbone-architecturen (bijvoorbeeld ResNet, Swin Transformer of ConvNeXt) zich verhouden qua compatibiliteit en prestaties binnen deze pretrainingsstrategieën. \\

Ook op het vlak van \gls{ssl} zijn er waardevolle alternatieven die verder onderzoek nodig hebben. FixMatch toonde in dit onderzoek al sterke resultaten, maar recentere methoden zoals FlexMatch, SoftMatch of \gls{uda} hebben mogelijk nog betere prestaties, zeker in combinatie met meer geavanceerde of domeinspecifieke augmentatietechnieken. Denk hierbij aan augmentaties die beter aansluiten bij sonarcontexten, zoals het toevoegen van realistische ruis, reflectievariaties of vervormingen die voortkomen uit onderwaterpropagatie.

Een bijzonder interessante piste is de combinatie van \gls{self-sl} pretraining en \gls{ssl} fine-tuning. Door eerst representaties te leren via een \gls{self-sl} benadering zoals \gls{byol} en deze vervolgens verder te verfijnen met een methode als FixMatch, kunnen de voordelen van beide strategieën worden gecombineerd. Experimenteel onderzoek moet uitwijzen of deze hybride aanpak leidt tot hogere nauwkeurigheid en robuustere detectieprestaties in situaties met een minimale hoeveelheid gelabelde data. \\

Verder kan onderzoek naar transfer learning binnen het sonardomein waardevolle inzichten opleveren. Modellen die getraind zijn op grootschalige ruwe sonardata uit verschillende toepassingen -- bijvoorbeeld onderwaternavigatie, structuurinspectie of archeologisch onderzoek -- kunnen geëvalueerd worden op hun vermogen om te generaliseren naar nieuwe omgevingen of sensorconfiguraties. Dit zou toelaten om universele sonarbackbones te ontwikkelen die met minimale aanpassing bruikbaar zijn in diverse domeinen. \\

Ten slotte is er nood aan onderzoek dat zich richt op de interactie tussen modelprestatie en labelselectie. Actief leren, waarbij het model zelf aangeeft welke samples het meest informatief zijn om te labelen, biedt hier een kansrijke uitbreiding. Door actief leren te combineren met \gls{self-sl} of \gls{ssl} technieken, kan mogelijk een nog efficiëntere annotatiestrategie ontwikkeld worden. Daarnaast verdient ook foutenanalyse meer aandacht: een diepgaand inzicht in de soorten fouten die modellen maken, kan bijdragen aan interpretatie, vertrouwen en verdere optimalisatie van detectieprestaties in kritieke toepassingen zoals defensie of onderwaterrobotica.