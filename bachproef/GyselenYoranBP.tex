%===============================================================================
% LaTeX sjabloon voor de bachelorproef toegepaste informatica aan HOGENT
% Meer info op https://github.com/HoGentTIN/latex-hogent-report
%===============================================================================

\documentclass[dutch,dit,thesis]{hogentreport}

\usepackage{lipsum} % For blind text, can be removed after adding actual content

\usepackage[acronym, toc]{glossaries}
\setglossarystyle{altlist}

\usepackage{float} % Place images where I want

\usepackage{tikz}
\usepackage{pgfplots}
\usepackage{pgfplotstable}
\usepackage{subcaption}

% Set TOC depth
\setcounter{tocdepth}{3}

% Add woordenlijst & acroniemen

% === Acroniemen ===
\newacronym{auv}{AUV}{Autonomous underwater vehicle}
\newacronym{mfls}{MFLS}{Multibeam forward-looking sonar}
\newacronym{rcnn}{R-CNN}{Region-based Convolutional Neural Networks}
\newacronym{yolo}{YOLO}{You Only Look Once}
\newacronym{ssd}{SSD}{Single-shot Detector}

% === Termen ===
\newglossaryentry{blindganger}
{
    name={Blindganger},
    text={blindganger},
    description={Explosief dat niet is afgegaan},
    plural={blindgangers},
    descriptionplural={Explosieven die niet zijn afgegaan}
}

\makenoidxglossaries

%% Pictures to include in the text can be put in the graphics/ folder
\graphicspath{{../graphics/}}

%% Source code highlighting
\usepackage[chapter]{minted}
\usemintedstyle{solarized-light}

%% Formatting for minted environments.
\setminted{%
    autogobble,
    frame=lines,
    breaklines,
    linenos,
    tabsize=4
}

%% Ensure the list of listings is in the table of contents
\renewcommand\listoflistingscaption{%
    \IfLanguageName{dutch}{Lijst van codefragmenten}{List of listings}
}
\renewcommand\listingscaption{%
    \IfLanguageName{dutch}{Codefragment}{Listing}
}
\renewcommand*\listoflistings{%
    \cleardoublepage\phantomsection\addcontentsline{toc}{chapter}{\listoflistingscaption}%
    \listof{listing}{\listoflistingscaption}%
}

%%---------- Document metadata -------------------------------------------------
\author{Yoran Gyselen}
\supervisor{Mevr. C. Teerlinck}
\cosupervisor{Mevr. S. Duyck}
\title[Efficiënte analyse door geavanceerde leertechnieken]%
    {Detecteren van Objecten in Sonardata met behulp van Semi- of Self-supervised Learning}
\academicyear{\advance\year by -1 \the\year--\advance\year by 1 \the\year}
\examperiod{2}
\degreesought{\IfLanguageName{dutch}{Professionele bachelor in de toegepaste informatica}{Bachelor of applied computer science}}
\partialthesis{false} %% To display 'in partial fulfilment'
\institution{Exail Robotics Belgium}

%% Add global exceptions to the hyphenation here
\hyphenation{back-slash}

%% The bibliography (style and settings are  found in hogentthesis.cls)
\addbibresource{bibliografie/bachproef.bib}            %% Bibliography file
\addbibresource{../voorstel/bibliografie/voorstel.bib} %% Bibliography research proposal
\defbibheading{bibempty}{}

%% Prevent empty pages for right-handed chapter starts in twoside mode
\renewcommand{\cleardoublepage}{\clearpage}

\renewcommand{\arraystretch}{1.2}

%% Content starts here.
\begin{document}

%---------- Front matter -------------------------------------------------------

\frontmatter

\hypersetup{pageanchor=false} %% Disable page numbering references
%% Render a Dutch outer title page if the main language is English
\IfLanguageName{english}{%
    %% If necessary, information can be changed here
    \degreesought{Professionele Bachelor toegepaste informatica}%
    \begin{otherlanguage}{dutch}%
       \maketitle%
    \end{otherlanguage}%
}{}

%% Generates title page content
\maketitle
\hypersetup{pageanchor=true}

%%=============================================================================
%% Voorwoord
%%=============================================================================

\chapter*{\IfLanguageName{dutch}{Woord vooraf}{Preface}}%
\label{ch:voorwoord}

%% TODO:
%% Het voorwoord is het enige deel van de bachelorproef waar je vanuit je
%% eigen standpunt (``ik-vorm'') mag schrijven. Je kan hier bv. motiveren
%% waarom jij het onderwerp wil bespreken.
%% Vergeet ook niet te bedanken wie je geholpen/gesteund/... heeft

Voor u ligt mijn bachelorproef over het gebruik van semi- en self-supervised learningtechnieken voor objectdetectie op sonarbeelden. Dit onderzoek richt zich op de vraag of dergelijke technieken het labelproces kunnen versnellen zonder significant verlies in nauwkeurigheid. Het doel is een efficiëntere methode ontwikkelen voor het verwerken van sonardata. \\

Mijn interesse in dit onderwerp ontstond vanuit een combinatie van mijn passie voor machine learning en de uitdagingen die ik tijdens mijn stage tegenkwam. Het verwerken en labelen van sonardata bleek een tijdrovend en arbeidsintensief proces te zijn. Dit gaf me het idee om innovatieve methoden te verkennen om dit te optimaliseren. Semi- en self-supervised learning boden een veelbelovende oplossing, en ik was benieuwd of deze technieken in de praktijk daadwerkelijk een verschil konden maken. \\

Het schrijven van deze bachelorproef was een enorm leerzaam, maar uitdagend proces, waarin ik veel heb bijgeleerd over machine learning en de praktische toepassingen ervan binnen de industrie. Dit onderzoek zou niet mogelijk zijn geweest zonder de steun en begeleiding van verschillende mensen, aan wie ik graag mijn dank wil uitspreken. \\

Allereerst wil ik mijn promotor, mevr. Chantal Teerlinck, bedanken voor de goede begeleiding, feedback en inzichten tijdens dit traject. Haar ervaring en ondersteuning hebben me geholpen om dit onderzoek in de juiste richting te sturen. Daarnaast wil ik mijn co-promotor, mevr. Stefanie Duyck, bedanken voor haar betrokkenheid, kennis en praktische inzichten vanuit de bedrijfswereld, wat een belangrijke meerwaarde vormde voor dit onderzoek. \\

Ook wil ik mijn dank uitspreken aan Exail Robotics Belgium voor de kans om mijn bachelorproef binnen hun organisatie uit te voeren. De toegang tot relevante -- en zeer waardevolle -- data en de begeleiding vanuit het team hebben een cruciale rol gespeeld in het realiseren van dit onderzoek. Tot slot wil ik mijn familie en vrienden bedanken voor hun steun en aanmoediging gedurende mijn studietraject. \\

Ik hoop dat deze bachelorproef een bijdrage kan leveren binnen het domein van objectdetectie op sonarbeelden en machine learning in het algemeen.
%%=============================================================================
%% Samenvatting
%%=============================================================================

\chapter*{Samenvatting}

Sinds de opkomst van krachtige AI- en deep learning-modellen is data uitgegroeid tot een essentiële en vaak beperkende factor in het ontwikkelingsproces. Waar eenvoudige modellen vaak kunnen volstaan met beperkte en eenvoudige datasets, vereisen complexere modellen -- zoals die voor objectdetectie -- steeds grotere en rijkere hoeveelheden gelabelde data. Dit vormt een belangrijk probleem in domeinen zoals sonarbeeldvorming, waar dergelijke datasets niet beschikbaar zijn als kant-en-klare bronnen en handmatige annotatie buitengewoon tijdsintensief en kostbaar is. Dit onderzoek richt zich daarom op de centrale vraag: hoe kunnen semi-supervised en self-supervised leermethoden het labelproces bij objectdetectie in sonardata versnellen, zonder significant verlies aan nauwkeurigheid? \\

Om deze vraag te beantwoorden is een experimenteel kader opgezet waarin drie benaderingen zijn onderzocht: een volledig supervised baseline gebaseerd op Faster R-CNN, een semi-supervised model met FixMatch, en een self-supervised strategie waarbij een BYOL-model wordt gepretraind en vervolgens gebruikt als backbone binnen een Faster R-CNN-architectuur. De experimenten zijn uitgevoerd op een publieke sonardataset bestaande uit 7600 gelabelde sonarbeelden. Voor de supervised baseline is het model getraind op verschillende hoeveelheden gelabelde data: 1\%, 5\%, 10\%, 50\% en 100\%. De bijbehorende mAP-scores tonen een sterke daling in nauwkeurigheid naarmate de hoeveelheid gelabelde data afneemt, met resultaten variërend van 0.7717 (100\%) tot slechts 0.2799 bij gebruik van 1\% van de data. \\

In het semi-supervised scenario is FixMatch toegepast met 5\% en 10\% gelabelde data, terwijl de resterende data werd gebruikt als ongelabelde input. Deze aanpak resulteerde in mAP-scores van respectievelijk 0.6649 en 0.6828, wat duidelijk betere prestaties zijn dan het supervised model op dezelfde labelniveaus. Voor het self-supervised model werd BYOL gepretraind op de volledige dataset zonder labels. De representaties die dit opleverde zijn vervolgens geïntegreerd in Faster R-CNN, waarbij opnieuw 5\% en 10\% van de data gelabeld werd gebruikt voor training. Deze benadering leverde de hoogste nauwkeurigheid binnen de lage-labelscenario's, met mAP-scores van respectievelijk 0.6452 en 0.7230.

\clearpage

De resultaten van dit onderzoek tonen aan dat zowel semi-supervised als self-supervised technieken effectief zijn in het verminderen van de afhankelijkheid van handmatig gelabelde data, terwijl de modelprestaties grotendeels behouden blijven. Met name self-supervised pretraining via BYOL blijkt zeer waardevol in situaties met beperkte gelabelde data. Deze bevindingen bieden praktische aanknopingspunten voor het ontwikkelen van efficiëntere workflows in sonarbeeldanalyse, en zijn relevant voor bredere toepassingen in domeinen waar gelabelde data schaars of moeilijk te verkrijgen is. Hoewel de resultaten veelbelovend zijn, is vervolgonderzoek nodig om de generaliseerbaarheid naar andere types sonardata of real-time toepassingen te evalueren.

%---------- Inhoud, lijst figuren, ... -----------------------------------------

\tableofcontents

\listoffigures

\listoftables

\listoflistings

\printnoidxglossary[type=main, toctitle=Woordenlijst]
\printnoidxglossary[type=acronym, toctitle=Acroniemen]

%---------- Kern ---------------------------------------------------------------

\mainmatter{}

%%=============================================================================
%% Inleiding
%%=============================================================================

\chapter{Inleiding}%
\label{ch:inleiding}

Objectdetectie heeft de afgelopen jaren enorme vooruitgang geboekt dankzij de opkomst van deep learning en de beschikbaarheid van grote, gelabelde datasets. In domeinen zoals computervisie, waar overvloedige trainingsdata gemakkelijk toegankelijk is, hebben deze technieken indrukwekkende prestaties bereikt. Echter beschikken niet alle vakgebieden over dergelijke datasets. In gespecialiseerde domeinen, zoals sonarbeeldanalyse, is gelabelde data schaars, wat het trainen van nauwkeurige detectiemodellen bemoeilijkt. Dit onderzoek richt zich op het verkennen van alternatieve leermethoden die deze afhankelijkheid van handmatige annotatie kunnen verminderen, zonder in te boeten op de prestaties van het model.

\section{Probleemstelling}%
\label{sec:probleemstelling}

Om een model te trainen dat met hoge precisie objecten in afbeeldingen kan herkennen en aanduiden, is een grote hoeveelheid gelabelde data nodig. Dit betekent dat naast de afbeeldingen zelf ook informatie over de positie van het object op de afbeelding beschikbaar moet zijn. Na de training is het de bedoeling dat het model deze informatie kan voorspellen op ongekende afbeeldingen. Sinds de opkomst van objectdetectie binnen het veld van machine learning zijn verschillende datasets openbaar beschikbaar gesteld die vrij gebruikt mogen worden om een dergelijk model te trainen. Dit geldt echter niet voor sonardata, om meerdere redenen. \\

Allereerst is objectdetectie op sonarbeelden een nicheprobleem. Hierdoor beschikken slechts weinig mensen over de kennis en expertise om zo’n dataset samen te stellen en, nog belangrijker, correct te annoteren. Daarnaast is er het – misschien nog grotere – probleem van de data zelf. Dit type gegevens kan niet eenvoudig met een camera worden verzameld; er is een gespecialiseerde sonarinstallatie voor nodig. Voor bedrijven is de aanschaf van zo’n installatie vaak weinig rendabel. Wel worden dergelijke systemen gebruikt voor militaire doeleinden, maar de data die hieruit voortkomt, is om veiligheidsredenen vrijwel altijd geclassificeerd. \\

De probleemstelling is dus tweeledig: er is weinig gelabelde sonardata voor objectdetectie beschikbaar, en het annoteren van een dergelijke dataset is moeilijk, tijdrovend en kostbaar.

\section{Onderzoeksvraag}%
\label{sec:onderzoeksvraag}

Om kosten en tijd te besparen, zou het dus ideaal zijn als er zo min mogelijk annotatie van de dataset nodig is. Bij supervised learning is dit echter nagenoeg onmogelijk, aangezien het model juist getraind wordt op basis van het verband tussen de afbeelding en de annotatie. Er bestaan echter veelbelovende alternatieven, zoals semi- en self-supervised learning, om dit probleem te overbruggen. Deze technieken maken gebruik van ongesuperviseerde data om representaties te leren en beperken zo de afhankelijkheid van gelabelde data. Moderne semi- en self-supervised methoden hebben indrukwekkende resultaten laten zien in domeinen zoals computer vision, maar hun toepassing op domeinspecifieke datasets, zoals sonar, is nog relatief onbekend terrein. \\

De hoofdvraag van dit onderzoek is daarom: Op welke manieren kan het gebruik van semi- of self-supervised learning het labelproces versnellen zonder een significant verlies in nauwkeurigheid?

\section{Onderzoeksdoelstelling}%
\label{sec:onderzoeksdoelstelling}

Het onderzoek verwacht aan te tonen dat semi- en self-supervised learning effectief kunnen worden ingezet om het labelproces bij sonarobjectdetectie aanzienlijk te versnellen. Door technieken zoals SimCLR, BYOL, Pseudo-Labeling en FixMatch te gebruiken voor pre-training, wordt verwacht dat het model sterke representaties leert van ongesuperviseerde sonardata, waardoor de behoefte aan grootschalige gelabelde datasets afneemt. Daarnaast zal een analyse inzicht geven in de minimale hoeveelheid gelabelde data die nodig is om vergelijkbare of betere prestaties te behalen dan met volledig supervised-learning methoden. \\

Dit resulteert in een efficiëntere en kosteneffectieve aanpak voor objectdetectie in sonarbeelden, zonder verlies van nauwkeurigheid, en biedt een waardevolle methodologie voor verdere toepassingen in domeinen waar gelabelde data schaars is.

\section{Opzet van deze bachelorproef}%
\label{sec:opzet-bachelorproef}

De rest van deze bachelorproef is als volgt opgebouwd: \\

In Hoofdstuk~\ref{ch:stand-van-zaken} wordt een overzicht gegeven van de stand van zaken binnen het onderzoeksdomein, op basis van een literatuurstudie. \\

In Hoofdstuk~\ref{ch:methodologie} wordt de methodologie toegelicht en worden de gebruikte onderzoekstechnieken besproken om een antwoord te kunnen formuleren op de onderzoeksvragen. \\

In Hoofdstuk~\ref{ch:data} begint het eigenlijke onderzoek. In deze eerste fase wordt een beschikbare dataset gepreprocessed voor gebruik in volgende hoofdstukken. \\

In Hoofdstuk~\ref{ch:implementatie} worden de verschillende modellen die vooraf uitgekozen zijn uit de verschillende categorieën geïmplementeerd in Python met behulp van Keras en TensorFlow. \\

In Hoofdstuk~\ref{ch:training-optimalisatie} worden de geïmplementeerde modellen getraind op de gepreprocesseerde dataset. Daarnaast worden de modellen geoptimaliseerd met behulp van hyperparameter tuning en andere technieken. \\

In Hoofdstuk~\ref{ch:evaluatie-modellen} worden de getrainde modellen geëvalueerd op basis van verschillende criteria. Er wordt een vergelijking gemaakt tussen -- onder andere -- de performantie van elk model en er wordt bepaald welk model de beste is. \\

In Hoofdstuk~\ref{ch:evaluatie-resultaten} wordt de praktische toepassing van de verschillende modellen geëvalueerd. Ook zullen enkele experts in sonaranalyse de bruikbaarheid van de resultaten beoordelen en aanbevelingen geven voor verdere verbeteringen. \\

In Hoofdstuk~\ref{ch:conclusie}, tenslotte, wordt de conclusie gegeven en een antwoord geformuleerd op de onderzoeksvragen. Daarbij wordt ook een aanzet gegeven voor toekomstig onderzoek binnen dit domein.
\chapter{Stand van zaken}
\label{ch:stand-van-zaken}

\section{Inleiding}

Objectdetectie in domeinspecifieke contexten zoals sonarbeeldvorming wordt vaak gehinderd door een gebrek aan gelabelde data. Traditioneel vereisen gesuperviseerde modellen grote hoeveelheden handmatig gelabelde gegevens om effectieve detectie en classificatie te leren. Semi-supervised en self-supervised learning bieden echter veelbelovende alternatieven door gebruik te maken van grote hoeveelheden ongesuperviseerde data om representaties te leren. Deze literatuurstudie bespreekt de huidige technieken en hun toepassing, met een specifieke focus op de unieke uitdagingen van sonardata.

% Objectdetectie in sonarafbeeldingen
\section{Objectdetectie in sonarafbeeldingen}

\subsection{Definitie en gebruik op sonarafbeeldingen}
\label{subsec:definitie-en-gebruik-op-sonarafbeeldingen}

Objectdetectie is een tak binnen het domein van computer vision dat gericht is op het identificeren en lokaliseren van objecten binnen beelddata (zoals foto's en video's). Dit wordt gebruikt in verschillende domeinen, zoals beveiligingssystemen (bv. om inbrekers te detecteren) of de medische wereld (bv. om tumoren op te sporen). Door de jaren heen is objectdetectie aanzienlijk geëvolueerd dankzij de vooruitgang in deep learning en de grote beschikbaarheid van datasets met beeldmateriaal. \autocite{He_2016} \\

Objectdetectie combineert twee belangrijke zaken in computer vision: objectlokalisatie en objectclassificatie. Objectlokalisatie bepaalt de positie van objecten, meestal in de vorm van \glspl{bounding_box} \autocite{Tompson_2015}, terwijl objectclassificatie bepaalt tot welke categorie een gedetecteerd object behoort. Samen geeft dit de mogelijkheid tot het herkennen van verschillende soorten objecten op één afbeelding.

Objectdetectie heeft vele toepassingen, ook in domeinen die misschien niet zo voor de hand liggend zijn. Één van deze specialisaties binnen de -- algemene -- objectdetectie is objectdetectie op sonardata. Dit domein is de laatste jaren erg gegroeid, vooral onder invloed van buitenlandse dreigingen. Zo wordt sonarobjectdetectie gebruikt voor het opsporen van mijnen in zee om ze later onschadelijk te kunnen maken. Naast detectie van mijnen wordt de techniek ook gebruikt voor verschillende soorten onderzoeken, zoals archeologisch en maritiem onderzoek. Bij deze verschillende toepassingen wordt natuurlijk telkens een kleine variatie op deze techniek gebruikt om telkens andere dingen op te sporen. \autocite{Wang_2024} \\

Traditioneel worden supervised learning methoden gebruikt voor objectdetectie. Voorbeelden van populaire architecturen binnen dit domein zijn onder andere Faster \gls{rcnn}, \gls{yolo} en \gls{ssd}. \autocite{Redmon_2016} Omdat dit supervised learning modellen zijn, presteren ze uitstekend bij voldoende gelabelde data. De annotatiekosten en tijdsinvestering vormen echter een grote belemmering, vooral bij complexe datasets zoals sonar. Sonardata vereist namelijk gespecialiseerde kennis voor het labelen, wat de annotatie nog uitdagender maakt. \autocite{Long_2015} \\

Deze supervised objectdetectietechnieken vallen onder te verdelen in grofweg twee grote categorieën. Enerzijds zijn er de zogenaamde \emph{single-shot detectors}, anderzijds heb je de \emph{two-stage detectors}. Deze categorieën zijn gebaseerd op hoeveel keer de afbeelding door het netwerk gaat. Bij single-shot detectors gaat de afbeelding slechts één keer door het netwerk, bij two-stage detectors -- logischerwijs -- twee keer. \autocite{Carranza_Garcia_2020} \\

Één van de eerste succesvolle toepassingen van deep learning binnen het domein van objectdetectie gebeurde in een bekend artikel van \textcite{Girshick_2013}. In dit artikel stelden de auteurs de \gls{rcnn}-architectuur voor. Deze veelbelovende architectuur behaalde een \gls{map} van 30\% meer dan de vorige topscore op een bekende publieke dataset (\acrshort{voc} 2012).

\subsubsection{Single-shot objectdetectie}

Zoals hierboven vermeld, is een single-shot detector een model waarbij de afbeelding slechts één keer door het netwerk gaat. Dit gebeurd door een \gls{cnn} te gebruiken dat zowel objectlocaties als bijbehorende classificaties voorspelt in één enkele \emph{pass}. Het voordeel van dit soort architectuur is dat ze zeer resource-efficiënt is, aangezien elke afbeelding slechts één keer behandeld wordt. Omwille van de efficiëntie, is deze architectuur dus uitermate geschikt voor real-time toepassingen zoals autonome voertuigen en bewakingssystemen. Een nadeel is echter dat het model niet altijd even precies is, aangezien het zowel locaties als klassen in één pass moet voorspellen. Vooral bij het detecteren van kleine objecten geeft dit een probleem. Om dit -- toch gedeeltelijk -- op te lossen, wordt soms gebruik gemaakt van meerdere schaalniveaus om objecten van verschillende groottes te detecteren, wat bijdraagt aan de robuustheid en precisie. \autocite{Carranza_Garcia_2020} \\

Zowel \gls{yolo} als \gls{ssd} zijn single-shot detectors. Deze architecturen worden verder besproken in latere secties hieronder.

\subsubsection{Two-stage objectdetectie}

Zoals de naam doet vermoeden, is two-stage objectdetectie een type objectdetectie waarbij de afbeelding twee keer door het netwerk gaat. Het resultaat is nog steeds een voorspelling van locaties en klassen, maar in plaats van in één keer -- zoals bij single-shot objectdetectie -- gebeurt de voorspelling nu in twee afzonderlijke stappen. Het resultaat van de eerste pass is een set van \emph{proposals} -- voorstellen -- en mogelijke locaties van objecten. Daarna zorgt de tweede pass voor een verfijning van deze voorstellen. Dit zorgt dan ook voor de uiteindelijke voorspellingen. Het voordeel van dit type objectdetectie is dat het veel preciezer is dan single-shot objectdetectie, aangezien de afbeelding twee keer geanalyseerd wordt. Het nadeel is dat deze approach veel resource-intensiever is. Het is dus zaak om een afweging te maken tussen de precisie van de voorspellingen en het verbruik van resources. \autocite{Carranza_Garcia_2020} \\

Over het algemeen wordt voor real-time applicaties single-shot objectdetectie gebruikt en voor cases waar de voorspellingen heel accuraat moeten zijn, wordt two-stage objectdetectie gebruikt. Een voorbeeld van een two-stage detector is de \gls{rcnn}-architectuur van \textcite{Girshick_2013}. Ook alle latere architecturen die hierop gebaseerd zijn, zijn two-stage detectors. Voorbeelden daarvan zijn Fast \gls{rcnn}, Faster \gls{rcnn} en Mask \gls{rcnn}. \autocite{Ren_2015}

\subsection{Meten van performantie binnen objectdetectie}

Gedurende dit onderzoek zal er gewerkt worden met verschillende soorten modellen en architecturen met elk hun sterktes en hun zwaktes. Het is de bedoeling dat al deze modellen vergeleken kunnen worden met elkaar om zo het best presterende model of de best presterende modellen te kunnen selecteren. Om dit te doen zijn er standaard metrieken nodig. Deze kunnen later (tijdens de trainingsfase) dan ook dienen als tussentijdse evaluatiemetriek om te zien hoe het model evolueert. Er zijn enorm veel verschillende metrieken om de performantie van objectdetectiemodellen te meten, maar twee van de bekendste en meest gebruikte zijn \gls{iou} en \gls{map}.

\subsubsection{Intersection over Union (IoU)}

\gls{iou} is een heel bekende metriek om de accuraatheid van de lokalisatie binnen objectdetectiemodellen te berekenen. Om de \gls{iou} te berekenen wordt gebruikt gemaakt van de echte \gls{bounding_box} en de voorspelde \gls{bounding_box}. Eerst wordt de overlappende oppervlakte (de doorsnede of \emph{intersection} in het Engels) van de twee \glspl{bounding_box} berekend. Daarna wordt de totale oppervlakte (de unie of \emph{union} in het engels) van de twee \glspl{bounding_box} berekend. \\

Door de doorsnede te delen door de unie krijgt men een verhouding van de overlappende oppervlakte tot de totale oppervlakte. Dit geeft een goede indicatie van hoe dicht de voorspelde \gls{bounding_box} bij de echte \gls{bounding_box} ligt. Een lagere \gls{iou}-score duidt op een betere prestatie, aangezien de voorspelde \gls{bounding_box} niet dan weinig afwijkt van de echte \gls{bounding_box}. \autocite{Rezatofighi_2019}

\begin{figure}[H]
    \centering
    \includegraphics[width=0.5\textwidth]{iou_equation.png}
    \caption[Voorstelling van IoU.]{\label{fig:iou_equation}Formule en visuele voorstelling van de \acrfull{iou}. \autocite{Rosebrock_2016}}
\end{figure}

\subsubsection{mean Average Precision (mAP)}

Een andere -- zeer bekende -- metriek is de \acrfull{map}. Ze beoordeelt de nauwkeurigheid van een model op basis van zowel \gls{precision} als \gls{recall}. Het wordt berekend door de gemiddelde precisie (of \emph{average precision}, AP) over alle klassen en \gls{iou} drempels te bepalen. AP wordt verkregen door de oppervlakte onder de precisie-recallcurve van een specifieke klasse te berekenen door hem te integreren, en \gls{map} is vervolgens het gemiddelde van deze waarden over alle klassen. Een hogere mAP-score duidt op betere prestaties van een objectdetectiemodel, omdat het aangeeft hoe goed het model objecten correct detecteert en classificeert. \autocite{Wang_2022}

\subsection{Typische uitdagingen bij sonarobjectdetectie}

Objectdetectie op sonarbeelden die gemaakt zijn onderwater wordt geconfronteerd met verschillende uitdagingen die de nauwkeurigheid en betrouwbaarheid van detecties beïnvloeden.

Ruis is één van de voornaamste obstakels bij sonarobjectdetectie. Onderwateromgevingen zijn inherent lawaaierig door factoren zoals luchtbellen, \glspl{thermocline} en biologische organismen, wat leidt tot significante ruis in sonarbeelden. Deze ruis bemoeilijkt het onderscheiden van echte objecten van artefacten typisch aan sonarbeelden, waardoor de betrouwbaarheid van de detecties afneemt. \autocite{Aubard_2024_Datasets} \\

Een andere uitdaging is de lage resolutie van sonarbeelden. In vergelijking met optische sensoren leveren sonars vaak beelden met beperkte detailweergave, wat de identificatie en classificatie van objecten moeilijker maakt. Deze beperking is vooral problematisch bij het detecteren van kleine of complexe objecten, waar detailniveau essentieel is voor nauwkeurige herkenning. \autocite{Lee_2018} \\

Daarnaast zorgen variërende omstandigheden in de onderwateromgeving voor extra complicaties. Factoren zoals veranderende waterdieptes, temperatuurverschillen, stromingen en de aanwezigheid van zwevende deeltjes kunnen de prestaties van sonarsystemen beïnvloeden. Deze dynamische omstandigheden kunnen leiden tot variaties in signaalsterkte en -kwaliteit, wat de consistentie van objectdetectie bemoeilijkt. \autocite{Valdenegro_Toro_2019} \\

Het overwinnen van deze uitdagingen vereist geavanceerde signaalverwerkingstechnieken en robuuste algoritmen die kunnen omgaan met ruis, lage resolutie en variabele omgevingsfactoren. Door voortdurende technologische innovaties en onderzoek kunnen de prestaties van sonarobjectdetectiesystemen worden verbeterd, wat leidt tot betrouwbaardere toepassingen in onderwateromgevingen.

\subsection{Overzicht van bestaande technieken}

Grofweg zijn er twee stromingen van objectdetectie op sonarafbeeldingen. Enerzijds zijn er de klassieke methoden en anderzijds zijn er de moderne deep learning-technieken. De klassieke methoden werden vooral gebruikt in een tijd waar grote, complexe neurale netwerken trainen onmogelijk was bij gebrek aan voldoende computerkracht, maar worden de dag van vandaag nog altijd gebruikt als pre-processing technieken voor de datasets waarmee de moderne neurale netwerken getraind worden. Deze klassieke methoden berusten enkel op statistische technieken om zo objecten in afbeeldingen te proberen detecteren. Specifiek zijn deze vooral gericht op het verbeteren van beeldkwaliteit en het onderscheiden van objecten van de achtergrond.

\clearpage

\subsubsection{Filtertechnieken}

Een voorbeeld van een klassieke methode zijn filtertechnieken. Deze worden toegepast om ruis in sonarafbeeldingen te verminderen en de beeldkwaliteit te verbeteren. Er bestaan immens veel verschillende soorten filters die elk geoptimaliseerd voor een specifiek doel. Een veelgebruikte filtermethode is het gebruik van adaptieve filters die zich aanpassen aan de lokale kenmerken van het beeld. Een voorbeeld hiervan is te vinden in een artikel van \textcite{Aridgides_1995}. Merk op dat dit inderdaad een relatief oude publicatie is, wat aantoont dat deze technieken al gebruikt werden toen deep learning-gebaseerde objectdetectie niet mogelijk was. \\

In dit artikel introduceren de auteurs een adaptieve filtertechniek die ontwikkeld is om mijnachtige doelen te onderscheiden van achtergrondruis in sonarbeelden. De filter onderdrukt achtergrondruis terwijl het de target behoudt. De procedure omvat vier stappen: het berekenen van een genormaliseerde gemiddelde target, het bepalen van de covariantiematrix van de achtergrondruis, het oplossen van normale vergelijkingen om een adaptief filter te verkrijgen en het toepassen van een 2D-filter op de gegevens. Dit algoritme bewijst dat, hoewel er geen gebruik gemaakt wordt van deep-learningtechnieken, ze toch complex kan zijn. De techniek heeft in verschillende testen prestaties geleverd die vergelijkbaar zijn met die van een ervaren sonaroperator. \\

Adaptieve filters worden ook vandaag de dag nog gebruikt, wat aangetoond wordt door een paper van \textcite{Lourey_2017}. Hierin wordt ook een filtertechniek beschreven die toegepast kan worden op \gls{cas} om interferentie van de directe transmissie en de echo van de target van elkaar te onderscheiden. Deze methode kan als effectieve pre-processing techniek gebruikt worden voor trainingsdata.

\subsubsection{Thresholding}

Naast filtering bestaan er nog andere klassieke methoden. Een \emph{straightforward}-aanpak is een simpele \emph{threshold}. Thresholding is een techniek waarbij pixelwaarden worden vergeleken met een bepaalde drempelwaarde om objecten van de achtergrond te scheiden. Een klassieke benadering is de Otsu-methode, die de interklassevariantie minimaliseert om een optimale drempelwaarde te bepalen. Deze methode wordt beschreven in een artikel van \textcite{Otsu_1979}.

\begin{figure}[H]
    \centering
    \begin{subfigure}{.5\textwidth}
        \centering
        \captionsetup{justification=centering}
        \includegraphics[width=0.9\linewidth]{img_pre_otsu.jpg}
        \caption[Afbeelding voor Otsu's thresholding]{Afbeelding voor Otsu's thresholding. \autocite{http//www.freephotos.lu/_2010}}
        \label{fig:img_pre_otsu}
    \end{subfigure}%
    \begin{subfigure}{.5\textwidth}
        \centering
        \captionsetup{justification=centering}
        \includegraphics[width=0.9\linewidth]{img_post_otsu.jpg}
        \caption[Afbeelding na Otsu's thresholding]{Afbeelding na Otsu's thresholding. \autocite{Pikez33_2010}}
        \label{fig:img_post_otsu}
    \end{subfigure}
    \caption[Afbeelding voor en na Otsu's thresholding]{Afbeelding voor en na Otsu's thresholding}
    \label{fig:imgs_otsu}
\end{figure}

Hoewel deze techniek oorspronkelijk is ontwikkeld voor visuele beelden, is deze ook toegepast op sonarafbeeldingen, zoals besproken in verschillende artikels, waaronder dat van \textcite{Yuan_2016} en dat van \textcite{Dimitrova_Grekow_2017}. Ondanks zijn simpliciteit kan deze techniek aanzienlijke verbeteringen teweegbrengen. Dit wordt onder andere aangehaald in een paper van \textcite{Komari_Alaie_2018}. Deze paper onderzoekt objectdetectie met passieve sonar in de Perzische Golf. Aangezien deze binnenzee ondiep is, is er sprake van een hoge hoeveelheid fouten tijdens de detectie. Een bepaald soort adaptieve thresholding-techniek kon de \gls{precision} van hun objectdetectiemodel met 24\% verbeteren.

\subsubsection{Edge detection}

Edge detection is een andere klassieke techniek die wordt gebruikt om de contouren van objecten in sonarafbeeldingen te identificeren. \autocite{Torre_1986} Een bekende methode is de Canny edge detector, die randen detecteert door het maximaliseren van de gradiëntgrootte. Deze techniek komt als beste uit de vergelijkende studie van \textcite{Awalludin_2022}. \\

De Canny edge detector werkt in meerdere stappen om nauwkeurige en robuuste contourdetectie te realiseren. De eerste stap is Gaussian blurring, waarbij het beeld wordt vervaagd om ruis te verminderen en kleine details die geen significante randen vormen te onderdrukken. Vervolgens wordt de gradiënt van het beeld berekend met behulp van Sobel-operatoren in zowel de horizontale als verticale richting, waardoor de randen worden geaccentueerd. Daarna wordt non-maximum suppression toegepast, waarbij alleen de sterkste randen worden behouden en omliggende pixels met lagere gradiëntwaarden worden onderdrukt. De laatste stap is hysteresis thresholding, waarbij twee drempelwaarden worden gebruikt: pixels met een gradiëntsterkte boven de hoge drempel worden als randen geclassificeerd, terwijl pixels onder de lage drempel worden genegeerd. Pixels met tussenliggende waarden worden alleen als rand beschouwd als ze verbonden zijn met een sterke randpixel. Dankzij deze gefaseerde aanpak is de Canny-methode effectief in het detecteren van duidelijke randen, zelfs in ruisgevoelige omgevingen zoals sonarafbeeldingen. \autocite{Ding_2001} \\

Ook edge detection wordt vandaag de dag nog gebruikt om een grote bijdrage te leveren aan bijvoorbeeld segmentatiemodellen. Het onderzoek van \textcite{Priyadharsini_2019} gebruikt gespecialiseerde edge detection als pre-processing voor de data naar een objectdetectiemodel gaat. \\

Deze klassieke technieken vormen de basis voor objectdetectie in sonarafbeeldingen en hebben bijgedragen aan de ontwikkeling van meer geavanceerde methoden. Ze blijven relevant, vooral in situaties waarin resources beperkt zijn of wanneer eenvoud en interpretatie van het model belangrijk zijn. Ze worden tot op de dag van vandaag gebruikt als pre-processing stap, bijvoorbeeld. Doordat computerkracht steeds goedkoper en meer beschikbaar werd, wordt tegenwoordig vaak gekozen voor deep learning-oplossingen voor deze problemen. Er zijn gespecialiseerde architecturen ontwikkeld om objectdetectie uit te voeren. Hieronder worden er enkele besproken.

\subsubsection{YOLO}

\acrfull{yolo} is een deep learning-gebaseerde architectuur voor objectdetectie dat bekend staat om zijn snelheid en efficiëntie. Het werd voor het eerst geïntroduceerd in een artikel van \textcite{Redmon_2016} en is sindsdien één van de populairste algoritmes in computervisie. In tegenstelling tot traditionele detectiemethoden, waar objecten in meerdere stappen geanalyseerd worden, verwerkt YOLO een afbeelding in één enkele \emph{pass} van het neurale netwerk. Dit zorgt ervoor dat real-time objectdetectie mogelijk is, waardoor het bijzonder geschikt is voor toepassingen zoals autonome voertuigen, videobewaking en \gls{ar}.

\begin{figure}[H]
    \centering
    \includegraphics[width=\textwidth]{yolo_architecture.png}
    \caption[Originele YOLO-architectuur.]{\label{fig:yolo_architecture}Schematische voorstelling van de originele YOLO-architectuur. \autocite{Redmon_2016}}
\end{figure}

\gls{yolo} gebruikt een \gls{cnn} om objecten direct te lokaliseren en classificeren, wat bijdraagt aan de hoge nauwkeurigheid en snelheid van het model. De eerste versie van \gls{yolo} gebruikt ImageNet om de eerste 20 convolutionele lagen te pre-trainen. Het model wordt daarna omgezet om detectie uit te voeren, aangezien de combinatie van convolutionele lagen en fully-connected-lagen de performantie verhoogt. \autocite{Redmon_2016} De interne werking van \gls{yolo} kan opgesplitst worden in verschillende fasen. \\

Eerst en vooral wordt de invoerafbeelding opgesplitst in een $S \times S$ raster (bv. $7 \times 7$). Elke cel van dat raster is daarna verantwoordelijk voor het detecteren van objecten waarvan het midden zich in dat vak bevindt. Voor elke cel voorspelt \gls{yolo} meerdere \glspl{bounding_box}. Zo'n voorspelling van een \gls{bounding_box} bevat telkens 5 parameters (cf. \ref{fig:bounding_box}): 

\begin{itemize}
    \item $x$ en $y$: de gecentreerde coördinaten van het object binnen de cel in het raster.
    \item $w$ en $h$: de breedte en hoogte van het object, genormaliseerd ten opzichte van de afbeelding.
    \item De \gls{confidence_score}: de waarschijnlijkheid dat er daadwerkelijk een object in de box zit.
\end{itemize}

Naast het voorspellen van de \glspl{bounding_box} voorspelt het model ook de klasse van het object (bv. auto, hond, persoon, \dots) en de \gls{confidence_score} van deze classificatie. Echter zijn er vaak meerdere \glspl{bounding_box} die hetzelfde object detecteren. Daarom gebruikt \gls{yolo} \gls{nms} om overbodige detecties te verwijderen. Ten slotte genereert \gls{yolo} een lijst met gedetecteerde objecten, hun locaties en de waarschijnlijkheid van hun klassen. \autocite{Diwan_2022} \\

\gls{yolo} is sneller dan traditionele methoden zoals Faster R-CNN omdat het objectdetectie beschouwt als een enkel regressieprobleem. Dit betekent dat het model direct van ruwe pixels naar detecties gaat, zonder een apart proces voor regio-voorstel en classificatie. \\

Sinds de introductie van \gls{yolo} in de paper van \textcite{Redmon_2016} heeft het model aanzienlijke verbeteringen en evoluties doorgemaakt. De oorspronkelijke versie, \gls{yolo}v1, legde de basis met een enkelvoudige doorvoer voor objectdetectie, maar had beperkingen in nauwkeurigheid, vooral bij kleine objecten. \gls{yolo}v2 (ook wel \gls{yolo}9000) werd kort na de introductie van het originele \gls{yolo}-model geïntroduceerd in een paper van \textcite{Redmon_2016_YOLOv2}. De nieuwe versie werd ontwikkeld om sneller en accurater te zijn dan het originele model. Ook kon deze versie meer verschillende klassen onderscheiden. Daarnaast werd er gebruik gemaakt van een andere \emph{backbone}, namelijk Darknet-19 (wat zelf een variant is van VGGNet). Dit zijn de belangrijkste veranderingen:

\begin{itemize}
    \item Gebruik van een nieuwe \gls{loss_functie} die beter geschikt is voor objectdetectie.
    \item Gebruik van \gls{batch_normalisatie} om accuraatheid en stabiliteit te verhogen.
    \item Trainen op zelfde afbeeldingen met een verschillende schaal, hierdoor wordt het model beter in het herkennen van kleine objecten.
    \item Gebruik van zogenaamde \emph{anchor boxes}: dit zijn vooraf gedefinieerde \glspl{bounding_box} die helpen bij het detecteren van objecten in een afbeelding. Tijdens het trainen leert het model welke van deze \emph{anchor boxes} het beste passen bij de werkelijke objecten in de afbeelding.
\end{itemize}

\gls{yolo}v3 werd geïntroduceerd in een paper van \textcite{Redmon_2018}. Het doel van deze iteratie was opnieuw het verbeteren van de accuraatheid en de snelheid. Dit deden de onderzoekers door opnieuw een andere architectuur te gebruiken als \emph{backbone}. Dit keer gebruikten ze Darknet-53 (een variant van ResNet). Daarnaast werden de \emph{anchor boxes} aangepast zodat ze verschillende vormen en maten hadden (in tegenstelling tot allemaal dezelfde vorm en maat in \gls{yolo}v2). Ook werden nog andere technieken toegepast om kleine objecten efficiënter en beter te kunnen detecteren. \\

De ontwikkeling van \gls{yolo} werd overgenomen door andere mensen, aangezien Joseph Redmond (de originele ontwikkelaar) na \gls{yolo}v3 de AI-community verliet. In een paper van \textcite{Bochkovskiy_2020} werd \gls{yolo}v4 geïntroduceerd. Hierin werd de efficiëntie verder verhoogd door verbeterde activatiefuncties en optimalisatietechnieken. \gls{yolo}v5, geïntroduceerd door Ultralytics, richtte zich op gebruiksvriendelijkheid en efficiëntere implementatie. \autocite{Jiang_2022} Nieuwere versies, zoals \gls{yolo}v7 en \gls{yolo}v8, blijven innoveren met hogere detectienauwkeurigheid, verbeterde verwerkingstijden en geavanceerdere architecturen, waardoor \gls{yolo} één van de meest gebruikte objectdetectiemodellen blijft in real-time toepassingen. \autocite{Terven_2023} \\

Ook heeft de architectuur veel potentie voor onderwaterobjectdetectie, zoals bij het opsporen van wrakken, onderzeese mijnen en mariene organismen. Dankzij de snelheid en efficiëntie van \gls{yolo} kunnen real-time detecties worden uitgevoerd, wat waardevol is voor \glspl{auv} en robots die in onbekende of gevaarlijke omgevingen opereren. Bovendien kunnen verbeterde versies van \gls{yolo}, zoals \gls{yolo}v5 en \gls{yolo}v8, met aangepaste architecturen en pre-processingtechnieken betere resultaten behalen op sonarbeelden. Door de voortdurende ontwikkeling van deep learning en sonarverwerking wordt \gls{yolo} steeds vaker ingezet voor geavanceerde onderwaterdetectie en navigatie. \autocite{Chen_2023}

\subsubsection{Faster R-CNN}

Een significante ontwikkeling in het domein van objectdetectie was de introductie van Faster \gls{rcnn} in een artikel van \textcite{Ren_2015}. Deze architectuur is een state-of-the-art model dat de snelheid en nauwkeurigheid van objectdetectie aanzienlijk heeft verbeterd. Het woord ``Faster'' in Faster \gls{rcnn} impliceert een duidelijke verbetering in snelheid ten opzichte van zijn voorgangers binnen de \gls{rcnn}-familie. Het feit dat Faster \gls{rcnn} zelfs meer dan tien jaar na zijn introductie nog steeds als referentiepunt wordt gebruikt in veel nieuwe objectdetectiepapers, benadrukt zijn blijvende relevantie en hoge mate van nauwkeurigheid.

\begin{figure}[H]
    \centering
    \includegraphics[width=0.5\textwidth]{faster_rcnn_architecture.png}
    \caption[Faster R-CNN-architectuur.]{\label{fig:faster_rcnn_architecture}Schematische voorstelling van de Faster \gls{rcnn}-architectuur. \autocite{Ren_2015}}
\end{figure}

\clearpage

Faster \gls{rcnn} is een zogenaamd two-stage detectiemodel, wat inhoudt dat het eerst regio’s in een afbeelding identificeert die mogelijk objecten bevatten, en deze vervolgens classificeert en verfijnt. Hoewel dit proces doorgaans minder snel is dan bij single-stage modellen, biedt het een grotere nauwkeurigheid in zowel lokalisatie als classificatie, wat verklaart waarom het model goed presteert in veel benchmarks. \\

De architectuur bestaat uit vier hoofdcomponenten:

\begin{enumerate}
    \item Een convolutionele backbone
    \item Het \gls{rpn}
    \item De \gls{roi} Pooling (of Align) laag
    \item De classificatie- en regressieheads
\end{enumerate}

De convolutionele backbone -- vaak een gepretraind netwerk zoals VGG-16 of ResNet -- is verantwoordelijk voor het extraheren van visuele kenmerken uit de invoerafbeelding. De keuze van backbone beïnvloedt zowel nauwkeurigheid als computationele resources: diepere netwerken zoals ResNet kunnen complexere patronen leren, maar vereisen ook meer resources. \\

Het \gls{rpn} vormt het hart van Faster \gls{rcnn}. Dit volledig convolutionele netwerk genereert potentiële objectlocaties direct vanuit de feature map. Het doet dit door op elke locatie meerdere zogenaamde \emph{ankers} te plaatsen: vooraf gedefinieerde \glspl{bounding_box} met diverse schalen en beeldverhoudingen. Voor elk anker voorspelt het \gls{rpn} een objectness score en voert het \gls{bounding_box} regressie uit. Hierdoor functioneert het \gls{rpn} als een soort attention-mechanisme dat de zoekruimte voor objectdetectie beperkt. De integratie van regio-voorstellen binnen het netwerk is de grootste innovatie van Faster \gls{rcnn} en maakt het model aanzienlijk sneller dan zijn voorgangers. \\

\gls{roi} Pooling (of \gls{roi} Align in nieuwere implementaties) zorgt ervoor dat regio’s van verschillende dimensies worden omgezet naar een vaste grootte, zodat ze verwerkt kunnen worden door de daaropvolgende fully connected lagen. Deze stap is essentieel voor de standaardisatie van input naar de classificatie- en regressieheads. \\

Tot slot bevat het model twee heads die parallel werken: het classificatiehead bepaalt de objectklasse binnen een regio en het regressiehead verfijnt de coördinaten van de \gls{bounding_box}. Deze dubbele aanpak zorgt voor nauwkeurige detectie en positionering van objecten. Ook wordt er gebruik gemaakt van \gls{nms}, die voorkomt dat meerdere \glspl{bounding_box} rond hetzelfde object worden behouden. \\

De originele \gls{rcnn} had aanzienlijke beperkingen, waaronder de trage verwerking van duizenden regio’s en een gefragmenteerd trainingsproces. Ook de afhankelijkheid van het traag werkende Selective Search-algoritme was een belangrijke bottleneck. Fast \gls{rcnn} verbeterde de snelheid door slechts één \gls{cnn}-pass over de volledige afbeelding te doen en daarna \gls{roi} pooling toe te passen. Toch bleef het afhankelijk van externe regio-voorstellen, wat de algehele efficiëntie beperkte. Faster \gls{rcnn} loste deze beperking definitief op. Door het gebruik van een geïntegreerd \gls{rpn} kon het model regio-voorstellen genereren als onderdeel van het netwerk zelf. Bovendien werd computationele efficiëntie verbeterd door convolutionele lagen te delen tussen het \gls{rpn} en de detectiecomponent (Fast \gls{rcnn}). Dankzij deze integratie is het model end-to-end trainbaar, met aanzienlijke winst in zowel snelheid als nauwkeurigheid.

\subsubsection{SSD}

Een ander state-of-the-art model dat populair is binnen academische onderzoeken, is \gls{ssd}. Deze architectuur werd voorgesteld in een artikel van \textcite{Liu_2016}. \gls{ssd} is een zogenaamde single-stage detector die een goede balans biedt tussen nauwkeurigheid en snelheid. \gls{ssd} detecteert objecten in één enkele pass door het netwerk, waarbij het gebruikmaakt van zogenaamde \emph{multiboxes} -- vooraf gedefinieerde rechthoeken die mogelijke objectlocaties voorstellen. Dit soort single-stage-architecturen zijn het gevolg van de vraag naar real-time beeldverwerking, zoals vereist in autonome voertuigen of slimme camera's. De geïntegreerde aanpak maakt \gls{ssd} aanzienlijk sneller. Hoewel two-stage modellen over het algemeen preciezer zijn, vooral bij kleine objecten, biedt \gls{ssd} een aantrekkelijk alternatief wanneer snelheid en eenvoud cruciaal zijn. \\

Het \gls{ssd}-algoritme werkt met één enkel neuraal netwerk dat tegelijk de objectcategorieën en de bijbehorende \glspl{bounding_box} voorspelt. Het netwerk verdeelt het beeld in meerdere \emph{default boxes} met verschillende schalen en beeldverhoudingen. Voor elke box voorspelt het de aanwezigheid van een object en past het de vorm van de box aan indien nodig. Belangrijk is dat \gls{ssd} gebruikmaakt van meerdere lagen (feature maps) uit het netwerk: lagen met een hoge resolutie detecteren kleine objecten, terwijl diepere lagen grotere objecten beter kunnen herkennen. Deze multi-scale aanpak is de sleutel tot de flexibiliteit van \gls{ssd} in het omgaan met objecten van verschillende groottes. \\

Een bijkomend voordeel van \gls{ssd} is dat het geen aparte stap nodig heeft voor het genereren van objectvoorstellen, wat tijd bespaart. Tegelijkertijd voert het netwerk voor elke \gls{bounding_box} twee voorspellingen uit: één voor de objectklasse en één voor de precieze locatie. Deze voorspellingen worden gedaan via kleine convolutionele filters. \\

De \gls{ssd}-architectuur bestaat uit een basisnetwerk voor feature extractie (zoals VGG16, ResNet of MobileNet), gevolgd door extra lagen die meerdere resoluties van het beeld analyseren. De oorspronkelijke implementatie gebruikte VGG16, maar varianten met lichtere of diepere netwerken zijn later ontwikkeld, afhankelijk van de toepassing. De keuze van het basisnetwerk bepaalt voor een groot deel de balans tussen nauwkeurigheid en gebruik van resources. \\

Omdat \gls{ssd} veel overlappende voorspellingen kan genereren, is een extra stap nodig om dubbele detecties te vermijden. Dit gebeurt via \gls{nms}, een techniek die alleen de meest waarschijnlijke voorspelling voor elk object behoudt en overlappende alternatieven verwijdert. \gls{nms} is hierbij essentieel voor het produceren van duidelijke en bruikbare detectieresultaten.

\begin{figure}[H]
    \centering
    \includegraphics[width=\textwidth]{ssd_architecture.png}
    \caption[Vergelijking YOLO- en SSD-architectuur.]{\label{fig:ssd_architecture}Vergelijking tussen de \gls{yolo}-architectuur en de \gls{ssd}-architectuur. \autocite{Liu_2016}}
\end{figure}

\gls{ssd} heeft een aantal belangrijke voordelen. Het is snel, eenvoudig en efficiënt in geheugengebruik en computationele resources. Dankzij het gebruik van meerdere feature maps kan het objecten van uiteenlopende groottes detecteren, en voor grote objecten is de nauwkeurigheid vaak uitstekend. Bovendien is de trainingsprocedure doorgaans sneller dan die van two-stage modellen. \\

Tegelijkertijd kent \gls{ssd} ook beperkingen. De nauwkeurigheid bij het detecteren van kleine objecten blijft een uitdaging. Verder kunnen er tijdens training veel ``achtergrondboxes'' aanwezig zijn, wat het moeilijker maakt voor het model om onderscheid te maken tussen achtergrond en relevante objecten. Ondanks het gebruik van verschillende schalen, kan het model moeite hebben met objecten die aanzienlijk kleiner of groter zijn dan wat het tijdens training heeft gezien. Ook kunnen de voorspelde locaties minder precies zijn dan bij sommige andere methoden.

\subsection{Specifieke toepassingen}

Objectdetectie met deep learning is de afgelopen jaren breed toepasbaar gebleken in uiteenlopende domeinen, mede dankzij de ontwikkeling van de krachtige modellen hierboven besproken. Elk van deze modellen heeft specifieke sterktes die hen geschikt maken voor verschillende praktische toepassingen, afhankelijk van de vereisten op het vlak van snelheid, nauwkeurigheid en computationele resources. \\

\gls{yolo} is ontworpen met real-time toepassingen in gedachten en blinkt uit in snelheid zonder een al te grote toegeving in nauwkeurigheid. Hierdoor wordt het model veel ingezet in domeinen waar onmiddellijke beslissingen noodzakelijk zijn. \autocite{Diwan_2022} In autonome voertuigen wordt \gls{yolo} bijvoorbeeld gebruikt om verkeersborden, voetgangers en obstakels in real-time te detecteren, wat cruciaal is voor veilige navigatie. Ook in bewakingssystemen komt \gls{yolo} goed tot zijn recht: het detecteert onmiddellijk personen of objecten in videobeelden en maakt snelle interventies mogelijk. Daarnaast zijn er toepassingen in de gezondheidszorg, zoals het lokaliseren van tumoren in medische beelden, en in de landbouw, waar het model ingezet wordt voor het herkennen van gewasziekten of onkruid. Het onderzoek van \textcite{Chen_2023} heeft aangetoond dat \gls{yolo} een praktisch nut biedt binnen onderwaterbeeldvorming. Daarnaast heeft het onderzoek van \textcite{Bochkovskiy_2020} aangetoond dat de ontwikkeling van het model ervoor heeft gezorgd dat het één van de beste keuzes is geworden voor objectdetectie. \\

\gls{ssd} biedt een compromis tussen snelheid en nauwkeurigheid en is daarmee eveneens geschikt voor real-time toepassingen, maar met iets betere detectieprestaties voor kleinere objecten dan \gls{yolo}. \autocite{Kumar_2020} \gls{ssd} wordt vaak ingezet in \gls{ar}-toepassingen, waar het accuraat identificeren van objecten in de fysieke omgeving essentieel is om digitale elementen er realistisch op af te stemmen. Ook in autonome voertuigen wordt \gls{ssd} gebruikt, bijvoorbeeld om bewegende objecten zoals voetgangers of andere auto's op tijd waar te nemen. In beveiligingstoepassingen helpt \gls{ssd} bovendien bij het monitoren van camerabeelden om ongewenste activiteiten automatisch te signaleren. Het onderzoek van \textcite{Ma_2020} heeft \gls{ssd} bijvoorbeeld gebruikt om afval te detecteren. Ook binnen onderwaterobjectdetectie is \gls{ssd} al enkele keren gebruikt, zoals in het onderzoek van \textcite{Jiang_2020}.

\clearpage

Faster \gls{rcnn}, ten slotte, is een model dat zich onderscheidt door zijn hoge nauwkeurigheid, maar in ruil daarvoor minder geschikt is voor real-time toepassingen door de hogere computationele requirements. Het model wordt daarom veel gebruikt in situaties waar precisie belangrijker is dan snelheid. Zo wordt Faster \gls{rcnn} ingezet in lucht- en satellietbeelden voor het identificeren van gebouwen, voertuigen of vegetatie. In de medische beeldanalyse wordt het model toegepast voor het nauwkeurig lokaliseren van afwijkingen zoals tumoren of letsels, wat essentieel is voor een betrouwbare diagnose. Ook in industriële toepassingen, zoals kwaliteitscontrole op productielijnen, helpt Faster \gls{rcnn} bij het detecteren van defecten met een hoge mate van precisie. Ten slotte heeft Faster \gls{rcnn} zijn effectiviteit al bewezen in onderwaterobjectdetectie, zoals in het onderzoek van \textcite{Wang_2023} of \textcite{Zeng_2021}. Het onderzoek van \textcite{Yulin_2020} heeft aangetoond dat het model ook op sonardata effectief is. Daarnaast is het model goed in het detecteren in kleine objecten (wat in sonardata vaak het geval is). Dit is gebleken uit het onderzoek van \textcite{Chen_2017}. \\

De keuze tussen deze modellen is sterk afhankelijk van de specifieke context waarin objectdetectie wordt toegepast. Terwijl \gls{yolo} en \gls{ssd} uitblinken in snelheid voor dynamische omgevingen, biedt Faster \gls{rcnn} de precisie die vereist is in kritieke of specialistische domeinen zoals geneeskunde en geospatiale analyse.

% Semi-supervised learning: principes en technieken
\section{Semi-supervised learning: principes en technieken}

\subsection{Definitie en waarde binnen objectdetectie}

Binnen het domein van machine learning bestaan er verschillende technieken om een model te trainen. Meestal wordt er gesproken van twee grote stromingen: supervised learning en unsupervised learning. Bij supervised learning wordt er gebruik gemaakt van een dataset en een uitkomst (hetgeen het model uiteindelijk moet kunnen voorspellen). Dit kan een label zijn of een bepaalde numerieke waarde. Belangrijk is dat zowel de input als de gewenste output gegeven zijn. Het model leert dus het verband tussen de twee. Bij unsupervised learning zijn er geen verwachte outputs. De volledige dataset wordt door het model gebruikt om patronen in te herkennen. Unsupervised learning wordt daarom ook meestal gebruikt om verkennende data-analyse uit te voeren. Echter hebben beide methoden enkele nadelen. Bij supervised learning is het traag en duur om alle data op een correcte manier te labelen. Unsupervised learning heeft dit probleem niet, maar heeft een beperkt aantal toepassingen en is minder accuraat. Een alternatief is \gls{ssl}, wat een compromis tussen zowel supervised als unsupervised learning is. \autocite{C_A_Padmanabha_Reddy_2018} \\

\gls{ssl} is een subdomein van machine learning waar gebruik gemaakt wordt van zowel gelabelde als ongelabelde data om modellen te trainen. Dit is bijzonder nuttig in situaties waarin het labelen van gegevens duur of tijdrovend is, zoals bij sonardata het geval is. \gls{ssl} bevindt zich tussen supervised learning (waar alle trainingsdata gelabeld zijn) en unsupervised learning (waar geen labels beschikbaar zijn). Door gebruik te maken van een kleine hoeveelheid gelabelde gegevens in combinatie met een grote hoeveelheid ongelabelde gegevens, kan een model beter generaliseren waardoor de prestaties verbeteren met minder menselijke annotatie-inspanning. \autocite{Hady_2013} \\

Zoals eerder vermeld, is \gls{ssl} een subdomein van machine learning, net zoals supervised en unsupervised learning. Het is dus niet één model of één architectuur, maar een waaier van verschillende algoritmen die soms op hele andere manieren werken. Wel hebben ze allemaal gemeen dat ze zowel gelabelde als ongelabelde data gebruiken. \\

Één van de belangrijkste technieken binnen \gls{ssl} is \emph{consistency regularization} (cf. infra), waarbij een model wordt aangemoedigd om consistente voorspellingen te maken voor kleine verstoringen van dezelfde ongelabelde input. \autocite{Fan_2022} Een ander veelgebruikt principe is \emph{pseudo-labeling} (cf. infra), waarbij het model zelf voorspellingen genereert voor ongelabelde gegevens en deze gebruikt als extra trainingsdata. \autocite{Lee_2013} \\

Daarnaast is er graph-based \gls{ssl}. Dit is een techniek die gebruik maakt van grafen (netwerken) om de relaties tussen gelabelde en ongelabelde data te modelleren en labelinformatie effectiever te verspreiden. In plaats van uitsluitend te vertrouwen op individuele gegevenspunten, gebruiken deze methoden de structuur van de dataset om aannames te maken over onbekende labels. Dit is vooral nuttig in situaties waarin de onderliggende data een natuurlijke connectiviteit vertoont, zoals sociale netwerken, biologische netwerken en tekstanalyses. \autocite{Song_2021} \\

Graph-based \gls{ssl}-modellen stellen de dataset voor als een graaf $G = (V, E)$ waarbij $V$ de knopen (datapunten) zijn (zowel gelabelde als ongelabelde gegevens) en $E$ de de gewogen randen (connecties) tussen knopen zijn, die de relatie of gelijkenis tussen de datapunten aangeven. Het basisidee is dat naburige knopen waarschijnlijk tot dezelfde klasse behoren, een principe dat bekend staat als \emph{label propagation}. Labels van bekende knopen (gelabelde data) worden hierbij iteratief verspreid naar naburige knopen op basis van de sterkte van de verbindingen. \autocite{Zhu_2005} \\

\gls{ssl} wordt toegepast in diverse domeinen, zoals beeld- en spraakherkenning, biomedische analyse en autonome systemen. Recente ontwikkelingen in deep learning hebben geleid tot geavanceerde \gls{ssl}-methoden, zoals FixMatch en MixMatch (cf. infra), die de prestaties aanzienlijk verbeteren door sterke data-augmentatie, efficiënter gebruik van ongelabelde data en de combinatie van andere \gls{ssl}-technieken. Onderzoek heeft aangetoond dat \gls{ssl} met slechts 10\% gelabelde data al bijna dezelfde prestaties kan bereiken als volledig gelabelde modellen. \autocite{Lucas_2022}

\subsection{Veelgebruikte SSL-methoden}

\acrfull{ssl} omvat verschillende methoden die gebruik maken van zowel gelabelde als ongelabelde data om de prestaties van machine learning-modellen te verbeteren. Veelgebruikte \gls{ssl}-methoden compenseren de beperkte beschikbaarheid van gelabelde data door patronen en structuren in ongelabelde data te benutten. \gls{ssl} wordt voor alle soorten doeleinden gebruikt: niet alleen in predictieve modellen, waarbij er vaak gebruik gemaakt wordt van pseudo-labeling en consistency regularization, wordt \gls{ssl} ingezet. Ook bij generatieve modellen wordt \gls{ssl} enorm veel gebruikt. Dit gaat dan bijvoorbeeld om \glspl{vae} en \glspl{gan}. Deze worden hierbij ingezet om aanvullende trainingsgegevens te genereren. Door al deze technieken te combineren, kunnen \gls{ssl}-methoden significante verbeteringen bieden voor taken zoals beeldherkenning, spraakverwerking en \gls{nlp}. \autocite{van_Engelen_2019} Hieronder worden enkele van de meest prominente technieken binnen \gls{ssl} besproken.

\subsubsection{Pseudo-labeling}

Pseudo-labeling is een belangrijke techniek binnen \gls{ssl} waarbij een model, getraind op een beperkte hoeveelheid gelabelde data, wordt ingezet om voorspellingen te doen op ongelabelde data. Deze voorspellingen, aangeduid als ``pseudo-labels'', worden vervolgens behandeld als echte labels, waardoor het model verder kan worden verfijnd met een uitgebreidere dataset. Dit proces wordt iteratief herhaald, zodat het model geleidelijk aan zijn prestaties verbetert door zowel de gelabelde als de pseudo-gelabelde data te gebruiken. \autocite{Lee_2013} \\

Het succes van pseudo-labeling is natuurlijk afhankelijk van de nauwkeurigheid van de gegenereerde pseudo-labels. Om de kwaliteit te waarborgen, wordt vaak een drempelwaarde ingesteld voor de voorspellingszekerheid: alleen voorspellingen die boven deze drempel uitkomen, worden als pseudo-labels geaccepteerd. Dit helpt het model om zich te concentreren op voorbeelden waarbij het relatief zeker is van de voorspelling, waardoor het risico op het leren van verkeerde informatie wordt verminderd. \autocite{Kage_2024} \\

Een belangrijk voordeel van pseudo-labeling is dat het effectief gebruikmaakt van grote hoeveelheden ongelabelde data, wat vooral nuttig is in domeinen zoals sonarbeeldvorming, waar het verkrijgen van gelabelde data duur en tijdrovend is. Verder zijn toepassingen van pseudo-labeling te vinden in verschillende gebieden, waaronder beeldherkenning, spraakverwerking en \gls{nlp}. \autocite{Min_2022} Recent onderzoek van \textcite{Ferreira_2023} heeft aangetoond dat pseudo-labeling een zeer goede performantie kan neerzetten, terwijl het de behoefte aan uitgebreide gelabelde datasets vermindert. \\

Desondanks kent pseudo-labeling ook uitdagingen. Als het model in een vroeg stadium onnauwkeurige pseudo-labels genereert, kan dit leiden tot het versterken van fouten, een fenomeen bekend als \emph{confirmation bias}. Om dit te voorkomen, worden technieken zoals \emph{curriculum learning} toegepast, waarbij het model eerst wordt getraind met de meest zekere pseudo-labels en geleidelijk aan minder zekere voorbeelden toevoegt naarmate de training vordert. \autocite{Cascante_Bonilla_2020}

\subsubsection{Consistency Regularization}

Een ander fundamenteel concept binnen \gls{ssl} is consistency regularization. Deze techniek wordt in verschillende algoritmen gebruikt om de betrouwbaarheid van het model te verbeteren. Het doet dit door het model te dwingen consistente voorspellingen te maken voor kleine variaties van dezelfde input. Het idee is gebaseerd op de veronderstelling dat een model robuust moet zijn tegen kleine verstoringen in de input, vooral wanneer de input geen gelabelde gegevens bevat. \\

Eerst wordt een ongelabeld voorbeeld $x_u$ aangepast met kleine verstoringen. Dit kan gaan om data-augmentatie (bv. willekeurige rotaties, verscherping, kleuraanpassingen, \dots), het toevoegen van ruis (bv. Gaussian noise), \dots Dit resulteert in twee versies van dezelfde input: het origineel $x_u$ en de verstoorde versie $x'_u$. Daarna voorspelt het model de kansverdeling van de klassen voor zowel de originele als de verstoorde input. \\

Een consistente voorspelling betekent dat beide kansverdelingen ``dicht'' bij elkaar moeten liggen. Om dit te garanderen wordt een gespecialiseerde \gls{loss_functie} gebruikt, zoals de \gls{kl}-divergentie \autocite{Hall_1987} of de \gls{mse}. Het model wordt getraind om deze \emph{loss} te minimaliseren, zodat het stabiele en robuuste voorspellingen leert maken, zelfs bij verstoringen. \\

Consistency regularization is enorm effectief (en wordt daarom ook veel gebruikt) omdat het ervoor zorgt dat het model gebruik maakt van de onderliggende structuur van ongelabelde data. Hierdoor verbetert de generalisatie, omdat het model minder gevoelig wordt voor kleine ruis en variaties. Daarnaast verhoogt de sample-efficiëntie, waardoor minder gelabelde data nodig is voor goede prestaties. \autocite{Fan_2022}

\subsubsection{FixMatch}

Beide voorgaande technieken -- zowel pseudo-labeling als consistency regularization -- zijn slechts concepten binnen het domein van \gls{ssl}. Dit wil daarom niet zeggen dat ze niet gebruikt worden in productie, maar vaak zijn ze slechts steunpilaren voor een grotere, complexere en effectievere architecturen. Hieronder wordt FixMatch -- een eerste voorbeeld die relevant is voor dit onderoek -- besproken. \\

FixMatch combineert pseudo-labeling met consistency regularization. Het werd geïntroduceerd door \textcite{Sohn_2020} en heeft bewezen indrukwekkende prestaties te leveren op benchmarks zoals CIFAR-10, waarbij met slechts 250 gelabelde voorbeelden een nauwkeurigheid van 94,93\% werd bereikt. De kernideeën achter FixMatch zijn gebaseerd op het principe dat een model consistente voorspellingen zou moeten maken, zelfs wanneer invoerdata wordt gewijzigd met verschillende vormen van data-augmentatie. FixMatch combineert twee belangrijke technieken binnen \gls{ssl}: pseudo-labeling en sterke versus zwakke augmentatie. \\

Voor een gegeven ongelabeld voorbeeld $x_u$ wordt eerst een zwak geaugmenteerde versie $x_u^w$ gegenereerd met eenvoudige transformaties zoals willekeurige spiegelingen en verschuivingen. Het model maakt een voorspelling $p\left(y|x_u^w\right)$, wat een kansverdeling over de mogelijke klassen oplevert. Als de hoogste voorspelde waarschijnlijkheid boven een drempel $\tau$ ligt wordt dit label als pseudo-label gebruikt. 

$$
\hat{y}_u = \arg \max p\left(y|x_u^w\right) \ \text{indien} \ \max p\left(y|x_u^w\right) > \tau
$$

Dit voorkomt dat onzekere voorspellingen als pseudo-labels worden gebruikt, waardoor ruis in de trainingsdata wordt verminderd. \\

Vervolgens wordt een sterk geaugmenteerde versie $x_u^s$ van dezelfde input $x_u$ gemaakt met agressieve augmentaties, zoals kleurvervormingen en Cutout (willekeurige afdekking van delen van de afbeelding). \autocite{DeVries_2017} Het model wordt getraind om dezelfde voorspelling $\hat{y}_u$ te maken op de sterk geaugmenteerde input $x_u^s$, waardoor het robuust wordt tegen verschillende variaties in de data. \\

Om FixMatch te trainen is er echter een gespecialiseerde \gls{loss_functie} nodig, die een combinatie is voor zowel de gelabelde als de pseudo-gelabelde data:

$$
L = L_s + \lambda_u L_u
$$

waarbij $L_s$ de gewone \gls{loss_functie} is voor de gelabelde data, $L_u$ de \emph{consistency loss} is voor pseudo-gelabelde data, gedefinieerd als de gemiddelde cross-entropy tussen het pseudo-label en de voorspelling op $x_{u'}^s$ en $\lambda_u$ een hyperparameter is die de balans tussen gelabelde en ongelabelde data regelt. \\

De hoge performantie van FixMatch heeft verschillende redenen: In tegenstelling tot traditionele pseudo-labeling, waarbij alle voorspellingen worden gebruikt, accepteert FixMatch alleen pseudo-labels als de modelvoorspelling een zekerheid boven een vooraf ingestelde drempel $\tau$ heeft. Dit voorkomt het versterken van foute pseudo-labels en verbetert de robuustheid van het model. Ook wordt het model gedwongen om dezelfde voorspelling te maken voor een input, ongeacht of deze zwak of sterk geaugmenteerd is. Dit helpt het model om stabiele, generaliseerbare representaties te leren. Daarnaast voorkomt het toepassen van een tweefasige-augmentatiestrategie dat het model te afhankelijk wordt van specifieke kenmerken in de data. Dit helpt overfitting te verminderen.

\subsubsection{MixMatch}

MixMatch integreert ook verschillende \gls{ssl}-technieken, waaronder data-augmentatie, pseudo-labeling en mixup. Het algoritme is een verbetering op de FixMatch-architectuur en werd geïntroduceerd door \textcite{Berthelot_2019}. In benchmarks heeft het hele goede prestaties geleverd op datasets zoals CIFAR-10 en STL-10. \\

Voor elk ongelabeld voorbeeld wordt het model meerdere keren toegepast om voorspellingen te genereren. Het gemiddelde van de voorspellingen wordt gebruikt om ruis te verminderen en betrouwbaardere pseudo-labels te verkrijgen. Dit helpt het model om robuustere pseudo-labels te creëren dan traditionele pseudo-labeling, waar slechts één voorspelling per voorbeeld wordt gebruikt. \\

Om ervoor te zorgen dat de pseudo-labels niet te onzeker zijn, wordt een \emph{softmax-temperature-scaling} toegepast om de kansverdeling scherper te maken. Dit betekent dat het model meer vertrouwen krijgt in de meest waarschijnlijke klassen, waardoor het pseudo-label effectiever wordt.

$$
q(y|x) = \frac{p(y|x)^{\frac{1}{T}}}{\sum_{y'} p(y'|x)^\frac{1}{T}}
$$

waarbij $T$ de temperatuurparameter is die de scherpte van de distributie regelt (kleinere waarden leiden tot scherpere distributies). \\

MixMatch maakt ook gebruik van \emph{mixup}, een data-augmentatietechniek waarbij synthetische trainingssamples worden gemaakt van paren bestaande trainingssamples en hun labels ($(x_1, y_1)$ en $(x_2, y_2)$). Merk op dat zowel de gelabelde als de pseudo-gelabelde samples gebruikt worden. 

\begin{align*}
    \tilde{x} & = \lambda x_1 + (1 - \lambda) x_2 \\
    \tilde{y} & = \lambda y_1 + (1 - \lambda) y_2
\end{align*}

Hierbij is $\lambda$ een gewichtsparameter, die willekeurig gekozen is uit een Beta-verdeling. Deze techniek zorgt voor soepelere beslissingsgrenzen en vermindert overfitting door het model te dwingen om robuuster te generaliseren. \autocite{Zhang_2017} \\

Omdat dezelfde input meerdere keren wordt geaugmenteerd en verwerkt, wordt het model getraind om consistente voorspellingen te maken voor deze variaties. Dit dwingt het model om stabiele representaties te leren, zelfs wanneer kleine veranderingen worden aangebracht in de inputdata. Net zoals FixMatch minimaliseert MixMatch een gecombineerde \gls{loss_functie} die zowel de gelabelde als de pseudo-gelabelde voorbeelden gebruikt (cf. supra). \autocite{Berthelot_2019}

% Self-supervised learning: principes en technieken
\section{Self-supervised learning: principes en technieken}

\subsection{Definitie en verschil met SSL}

Anders dan bij \gls{ssl} is er bij \gls{self-sl} helemaal geen nood aan labels. Deze creëert het algoritme namelijk zelf tijdens een pre-training fase. \Gls{self-sl} behoort echter niet helemaal tot het domein van unsupervised learning, hoewel het gebruik maakt van verschillende groeperings- en clusteringmethoden. Het uiteindelijke model maakt namelijk gebruik van -- door de pre-training -- gelabelde data. Deze techniek zorgt ervoor dat er veel complexere modellen getraind kunnen worden zonder een gigantische hoeveelheid aan data. Enkele nadelen zijn wel dat de techniek een grote hoeveelheid computerkracht nodig heeft en een lagere accuratie heeft dan supervised learning. \autocite{Gui_2024} \\

Het fundamentele verschil tussen \gls{self-sl} en \gls{ssl} ligt in de manier waarop ze omgaan met labels. In \gls{ssd} zijn er externe, door mensen geannoteerde labels aanwezig, zij het in beperkte mate. \Gls{self-sl} genereert daarentegen volledig zijn eigen labels zonder externe annotaties. Hierdoor wordt \gls{self-sl} soms beschouwd als een vorm van unsupervised learning, maar met expliciet gedefinieerde pretext-taken om de onderliggende structuren in data beter te benutten. \\

Ondanks deze nadelen blijken verschillende populaire technieken succesvol binnen het domein van computer vision.

\subsection{Werking}

\Gls{self-sl} probeert dus structuur, patronen of representaties te leren uit ruwe data door artificiële taken te formuleren die als ``surrogaat''-targets dienen. De fundamentele werking van self-supervised learning steunt op het concept van pretext-taken: eenvoudige, automatisch afleidbare taken die een model dwingen om zinvolle representaties te leren. Voorbeelden van zulke taken zijn het voorspellen van het ontbrekende deel van een afbeelding (\emph{inpainting}), het reconstrueren van een permutatie van patches (\emph{jigsaw puzzles}), of het bepalen van de relatieve positie tussen twee afbeeldingsfragmenten. Hoewel het model op deze taken niet direct getraind wordt om het uiteindelijke doel (zoals classificatie of detectie) te bereiken, leert het wél onderliggende semantische structuren en contexten herkennen -- wat waardevol blijkt bij finetuning voor downstream-taken. \\

In recente jaren is er een verschuiving ontstaan richting contrastieve en predictieve methodes binnen \gls{self-sl}. Contrastive learning probeert gelijkaardige samples dichter bij elkaar te brengen in de embedding-ruimte en verschillende samples uit elkaar te duwen. \autocite{Oord_2018} Bekende frameworks zoals \gls{simclr} en \gls{moco} gebruiken deze techniek. \gls{byol}, daarentegen, doet afstand van expliciete negatieve paren en vertrouwt op twee netwerken -- een online netwerk en een target netwerk -- die via momentum-updates met elkaar samenwerken (cf. infra). Het model leert door representaties van verschillende augmentaties van hetzelfde beeld op elkaar af te stemmen. Hierdoor leert het betekenisvolle representaties zonder dat negatieve voorbeelden nodig zijn. \\

Het einddoel van \gls{self-sl} is het trainen van een krachtige feature-extractor -- of backbone -- die generaliseerbare en contextuele representaties levert. Deze backbone kan daarna gebruikt worden in taken zoals objectdetectie, segmentatie of classificatie. In veel toepassingen, zoals bij het pretrainen van een convolutioneel neuraal netwerk voor gebruik in Faster \gls{rcnn}, is \gls{self-sl} een robuust alternatief voor pretraining op volledig gelabelde datasets. Daarmee maakt \gls{self-sl} het mogelijk om beter gebruik te maken van grote hoeveelheden ruwe data -- een cruciale troef in domeinen waar gelabelde data schaars is.

\subsection{Veelgebruikte Self-SL methoden}

In de context van vooruitgang in \gls{ssl} zijn er verschillende praktische modellen ontwikkeld die de eerder besproken theoretische principes -- zoals contrastive learning en representatieleren zonder gelabelde data -- succesvol toepassen op grootschalige visuele data. Deze modellen onderscheiden zich niet alleen door hun architecturale keuzes, maar ook door hoe ze omgaan met negatieve samples, augmentatiestrategieën en het gebruik van momentum-updates. In deze sectie worden enkele invloedrijke en veelgebruikte \gls{ssl}-methoden besproken in realistische settings, met een focus op \acrfull{simclr}, \acrfull{moco} en \acrfull{byol}. Elk van deze modellen illustreert op unieke wijze hoe self-supervised representaties kunnen worden geleerd met minimale supervisie, en vormt daarmee een belangrijke bouwsteen in moderne computer vision toepassingen.

\subsubsection{SimCLR}

Een bekende \gls{self-sl}-techniek is \acrshort{simclr}. Deze afkorting staat voor \acrlong{simclr}. Met andere woorden maakt deze techniek dus gebruik van \emph{contrastive learning} om visuele representaties te leren zonder de noodzaak van gelabelde data. Het werd ontwikkeld door onderzoekers van Google Brain en gepresenteerd in een paper van \textcite{Chen_2020}. Het doel van \gls{simclr} is om een neuraal netwerk zodanig te trainen dat het visuele representaties van afbeeldingen leert door contrastieve relaties te benutten.

Om een model contrastieve representaties te laten leren, worden er van elke invoerafbeelding eerst twee willekeurige transformaties gemaakt. Deze transformaties kunnen bestaan uit verschillende dingen, waaronder:

\begin{itemize}
    \item Willekeurig bijsnijden en schalen
    \item Kleurveranderingen zoals de helderheid en contrast aanpassen
    \item Toepassen van blurring
    \item Rotatie of horizontale spiegeling
\end{itemize}

Deze transformaties zorgen ervoor dat het model leert om dezelfde afbeelding te herkennen, ongeacht variaties in uiterlijk. Na de augmentaties worden de twee versies van de afbeelding door een encoder gestuurd, meestal een \gls{cnn} (zoals een ResNet). Dit netwerk zet de invoerafbeeldingen om in zogenaamde \emph{feature vectors} die de kernkenmerken van de afbeelding representeren. \\

De gegenereerde feature vectors worden vervolgens door een \emph{projection head} gestuurd. Dit is een klein neuraal netwerk dat de feature vector transformeert naar een ruimte waarin de contrastieve vergelijking plaatsvindt (latente ruimte). Dit projection head bestaat meestal uit een paar  fully connected of dense-lagen en wordt na training weggegooid, omdat alleen de encoder nodig is voor downstream taken (zoals beeldclassificatie, objectdetectie en semantische segmentatie). \autocite{Gupta_2022} \\

Het doel van SimCLR is om representaties van verschillende augmentaties van dezelfde afbeelding dichter bij elkaar te brengen en representaties van verschillende afbeeldingen verder uit elkaar te duwen. Dit gebeurt met behulp van de \gls{nt-xent} loss. De \gls{loss_functie} wordt berekend zodat positieve paren (twee augmentaties van dezelfde afbeelding) een hoge gelijkenis hebben en negatieve paren (verschillende afbeeldingen in de \gls{batch}) een lage gelijkenis. De cosinusgelijkheid wordt vaak gebruikt om de afstand tussen de verschillende vectoren te meten. Daarnaast beïnvloedt de temperatuurparameter $\tau$ in de \gls{nt-xent} loss  hoe streng de loss reageert op verschillen in gelijkenis tussen paren.

\subsubsection{MoCo}

Een andere veelgebruikte \gls{self-sl}-techniek is \gls{moco}. Deze methode maakt gebruik van contrastief leren om visuele representaties te leren zonder de noodzaak van gelabelde data. Het werd geïntroduceerd in een paper van \textcite{He_2019} en heeft sindsdien aanzienlijke aandacht gekregen binnen het domein van de computer vision. \gls{moco} introduceert verschillende innovatieve technieken zoals het \emph{momentum update}-mechanisme en een dynamisch woordenboek om contrastief leren te verbeteren. Ook wordt er een gespecialiseerde \gls{loss_functie} gebruikt die de basis vormt van het leerproces. Dankzij deze strategieën kan \gls{moco} superieure representaties leren zonder gelabelde data, wat het een krachtige \gls{self-sl}-techniek maakt. \\

Het \gls{moco}-framework bestaat uit meerdere essentiële componenten die samenwerken om effectieve visuele representaties te leren. De \emph{query encoder} ($f_q$) is verantwoordelijk voor het omzetten van een \emph{query sample} (zoals een geaugmenteerde versie van een afbeelding) in een \emph{feature vector}. De parameters van deze encoder worden bijgewerkt via standaard backpropagation, waarbij de optimalisatie wordt gestuurd door een contrastieve \gls{loss_functie}. \autocite{Sowe_2025} \\

Daarnaast bevat \gls{moco} een \emph{momentum key encoder} ($f_k$) die wordt gebruikt om \emph{keys} (bijvoorbeeld geaugmenteerde weergaven uit eerdere \glspl{mini_batch}) om te zetten in \emph{feature vectors}. Een cruciale innovatie van \gls{moco} is het \emph{momentum update}-mechanisme. In plaats van direct via backpropagation te worden bijgewerkt, worden de parameters van de \emph{key encoder} ($\theta_k$) aangepast als een voortschrijdend gemiddelde van de \emph{query encoder} ($\theta_q$). Dit gebeurt volgens de volgende formule:

$$
\theta_k \rightarrow m\theta_k + (1 - m)\theta_q
$$

Hierbij is $m$ de \emph{momentumcoëfficiënt}, die doorgaans dicht bij 1 ligt (bv. 0,999). Dit zorgt voor een geleidelijke en stabiele update van de key encoder, waardoor de representaties van de keys na verloop van tijd minder variëren. Een stabieler woordenboek van negatieve samples is cruciaal voor contrastief leren, omdat het helpt om robuuste en onderscheidende kenmerken te leren. \\

Om een grote en gevarieerde set van negatieve samples te behouden, maakt \gls{moco} gebruik van een dynamisch woordenboek dat wordt beheerd via een FIFO (First-In, First-Out) wachtrij. Dit mechanisme zorgt ervoor dat de grootte van het woordenboek niet wordt beperkt door de mini-batchgrootte, zoals bij traditionele contrastieve leermethoden het geval is.

\begin{itemize}
    \item Wanneer een nieuwe mini-batch wordt verwerkt, worden de bijbehorende representaties aan het einde van de wachtrij toegevoegd (enqueue).
    \item Tegelijkertijd worden de oudste representaties uit de wachtrij verwijderd (dequeue).
\end{itemize}

Dit ontkoppelt het aantal negatieve samples van de batchgrootte, waardoor \gls{moco} kan werken met veel grotere negatieve sets zonder dat dit een grote hoeveelheid GPU-geheugen vereist. Een groter en gevarieerder aantal negatieve samples helpt het model om betere representaties te leren, omdat de contrastieve taak uitdagender wordt. Dit dwingt het model om robuustere en meer onderscheidende kenmerken te ontwikkelen. \\

\gls{moco} maakt gebruik van de InfoNCE loss, een contrastieve verliesfunctie die het model leert om:

\begin{enumerate}
    \item De gelijkenis tussen een query en zijn bijbehorende positieve key te maximaliseren (dit zijn verschillende augmentaties van dezelfde afbeelding).
    \item De gelijkenis tussen de query en negatieve keys te minimaliseren (deze representeren andere afbeeldingen in de wachtrij).
\end{enumerate}

De wiskundige formulering van de InfoNCE loss is als volgt:

$$
L_q = - \log \frac{\exp{(q \cdot k^+ / {\tau})}}{\sum_{i=0}^{K}\exp{(q \cdot k_i / {\tau})}}
$$

Hierbij geldt:

\begin{itemize}
    \item $q =$ de gecodeerde query
    \item $k^+ =$ de gecodeerde positieve key
    \item $k_i =$ alle keys in het woordenboek (inclusief $k^+$ en $K$ negatieve keys)
    \item $\tau =$ de temperatuurparameter, die de scherpte van de loss controleert  
\end{itemize}

De InfoNCE loss vormt de kern van het leerproces van \gls{moco} en stelt het model in staat om een embedding space te creëren waarin augmentaties van dezelfde afbeelding dicht bij elkaar liggen en representaties van verschillende afbeeldingen ver uit elkaar worden geplaatst. Dit proces, bekend als \emph{instance discrimination}, helpt het model om semantisch betekenisvolle kenmerken te leren zonder gelabelde data.

\subsubsection{BYOL}

Naast de contrastieve methoden die veelal gebruikt worden binnen het domein van \gls{self-sl} is er sinds enkele jaren ook een ander alternatief. \gls{byol} is ontwikkeld bij Google DeepMind en werd geïntroduceerd in een paper van \textcite{Grill_2020}. \gls{byol} is gebaseerd op het idee dat een model kan leren door zijn eigen representaties te gebruiken als leersignaal. Dit gebeurt door twee verschillende, geaugmenteerde versies van hetzelfde inputbeeld te genereren, en het model leert vervolgens om de representatie van de ene versie te voorspellen op basis van de andere. In tegenstelling tot contrastieve leermethoden -- die gebruikmaken van zowel positieve als negatieve paren -- werkt \gls{byol} volledig zonder negatieve voorbeelden. Terwijl contrastieve methoden proberen representaties van verschillende beelden uit elkaar te duwen, focust \gls{byol} zich enkel op het dichter bij elkaar brengen van representaties van dezelfde onderliggende input. Deze aanpak vermindert de kans op instabiliteit en maakt het model robuuster voor variaties in data-augmentatie. \\

Het leerproces in \gls{byol} is gebouwd op twee samenwerkende netwerken: een \emph{online netwerk} en een \emph{target netwerk}. Het online netwerk probeert de representatie te voorspellen die door het target netwerk is gegenereerd. Het target netwerk zelf wordt niet direct getraind via \gls{backpropagation}, maar geüpdatet via een \gls{ema} van de parameters van het online netwerk. Deze samenwerking tussen de twee netwerken zorgt voor een geleidelijke en stabiele leeromgeving. \\

Zoals hierboven vermeld, bestaat de \gls{byol}-architectuur dus twee identieke netwerken qua structuur: het online netwerk en het target netwerk. Het online netwerk bevat drie componenten:

\begin{enumerate}
    \item \textbf{Encoder ($f_\theta$):} extraheert een representatie uit een geaugmenteerd inputbeeld.
    \item \textbf{Projector ($g_\theta$):} projecteert deze representatie naar een lagere-dimensionale vectorruimte.
    \item \textbf{Predictor ($q_\theta$):} probeert de representatie van het target netwerk te voorspellen.
\end{enumerate}

Het target netwerk bevat alleen de encoder en projector, zonder predictor. Het wordt bijgehouden als een langzame, voortschrijdende kopie van het online netwerk, waardoor het stabielere leerdoelen biedt. Tijdens training worden twee verschillende augmentaties van eenzelfde afbeelding verwerkt: één door het online netwerk en de andere door het target netwerk. Het doel is dat de output van de predictor van het online netwerk overeenkomt met de projectie van het target netwerk. \\

De \gls{loss_functie} voor \gls{byol} is gebaseerd op de \gls{mse} tussen L2-genormaliseerde vectoren: de voorspelling van het online netwerk en de projectie van het target netwerk. \autocite{Laarhoven_2017} Voor een enkel augmentatiepaar $(v, v')$ wordt de loss als volgt berekend:

$$
L_{\theta, \xi} = \left\Vert \text{normalize}\left(q_\theta\left(z_\theta\right)\right) - \text{normalize}\left(z'_\xi\right) \right\Vert^2_2 = 2 - 2 \cdot \left\langle \text{normalize}\left(q_\theta\left(z_\theta\right)\right), \text{normalize}\left(z'_\xi\right) \right\rangle
$$

Data-augmentatie speelt een centrale rol in \gls{byol}. Door willekeurige transformaties toe te passen (zoals crops, kleurvervormingen, flips, \dots), ontstaan twee verschillende maar inhoudelijk gelijke views van hetzelfde beeld. Deze variatie dwingt het model om representaties te leren die geen rekening houden met oppervlakkige veranderingen en gericht zijn op de semantische kern van het beeld. Interessant is dat \gls{byol} minder gevoelig is voor de specifieke keuze van augmentaties dan contrastieve methoden zoals \gls{simclr}. Het model blijft effectief presteren, zelfs wanneer bepaalde augmentaties worden weggelaten, wat wijst op een grotere robuustheid.

% Vergelijking van relevante methoden
\section{Vergelijking van relevante methoden}

In dit onderzoek zijn verschillende leermethoden onderzocht binnen drie bredere categorieën: supervised learning, \gls{ssl}, en \gls{self-sl}. Binnen elke categorie zijn meerdere representatieve modellen geëvalueerd op basis van hun geschiktheid voor objectdetectie in sonardata, met aandacht voor aspecten zoals precisie, robuustheid, data-efficiëntie, en computationele vereisten. 

\subsection{Gesuperviseerde architecturen}

Binnen het domein van supervised objectdetectie zijn \gls{yolo}, Faster \gls{rcnn} en \gls{ssd} drie toonaangevende architecturen met uiteenlopende sterktes en toepassingsgebieden. Alle drie zijn ontworpen om objectdetectie mogelijk te maken in volledig gelabelde datasets, maar ze verschillen aanzienlijk in hun aanpak van snelheid, nauwkeurigheid en architecturale complexiteit. \\

\gls{yolo} is een single-stage detector die objectdetectie als een regressietaak benadert, waarbij alle voorspellingen in één stap worden gedaan. Dit resulteert in zeer snelle inferentie, waardoor \gls{yolo} bijzonder geschikt is voor real-time toepassingen. De keerzijde is echter dat deze snelheid gepaard gaat met een lagere precisie bij het lokaliseren van kleine of dicht op elkaar liggende objecten -- een mogelijke beperking bij complexere sonarbeelden. \\

Faster \gls{rcnn} daarentegen maakt gebruik van een tweefasige detectiestructuur. Eerst genereert het \gls{rpn} potentiële objectlocaties, waarna een tweede netwerk verantwoordelijk is voor classificatie en verfijning van de \glspl{bounding_box}. Hoewel deze aanpak leidt tot een hogere inferentietijd, is de nauwkeurigheid doorgaans beter, vooral in complexe en ruisgevoelige beelden. Dit maakt Faster \gls{rcnn} conceptueel aantrekkelijker voor toepassingen waarin precisie belangrijker is dan snelheid, zoals bij de analyse van onderwaterbeelden via sonar. \\

\gls{ssd} neemt eveneens een single-shot benadering, maar introduceert verbeteringen zoals meerdere schalen en ankerboxen, wat de nauwkeurigheid enigszins verhoogt ten opzichte van vroege \gls{yolo}-versies. \gls{ssd} biedt een evenwicht tussen snelheid en precisie, maar is nog steeds gevoeliger voor objecten met onregelmatige vormen of weinig contrast -- karakteristiek voor sonaromgevingen.

\begin{table}[H]
    \centering
    \begin{tabular}{llll}
        \toprule
        \textbf{Eigenschap} & \textbf{\acrshort{yolo}} & \textbf{Faster \acrshort{rcnn}} & \textbf{\acrshort{ssd}} \\
        \midrule
        Architectuur            & Single-shot   & Two-stage     & Single-shot \\
        Nauwkeurigheid          & Gematigd      & Hoog          & Gemiddeld \\
        Snelheid                & Zeer snel     & Traag         & Snel \\
        Robuustheid bij ruis    & Beperkt       & Hoog          & Matig \\
        \bottomrule
    \end{tabular}
    \caption[Vergelijking supervised modellen]{\label{tab:comparison_supervised_models} Tabel met een simpele vergelijking tussen de onderzochte supervised modellen binnen dit onderzoek.}
\end{table}

\subsection{Semi-supervised architecturen}

FixMatch en MixMatch vormen binnen \gls{ssl} twee representatieve methoden die elk op een eigen manier trachten om het leerproces te verbeteren met behulp van een beperkte hoeveelheid gelabelde data en een grotere hoeveelheid ongesuperviseerde samples. Beide modellen zijn gebaseerd op het idee van consistency regularization -- het principe dat het model consistente voorspellingen moet doen voor verschillende transformaties van dezelfde input -- maar implementeren dit op fundamenteel verschillende manieren. \\

MixMatch combineert meerdere technieken in één framework: het genereert soft pseudo-labels via averaging, past entropy minimization toe om de outputdistributie te verscherpen, en gebruikt een MixUp-strategie om input en labelparen te interpoleren. Deze combinatie kan krachtige resultaten opleveren, vooral bij complexe data, maar introduceert ook aanzienlijke modelcomplexiteit en vereist precieze tuning van hyperparameters om optimale prestaties te behalen. Dit kan een uitdaging vormen in domeinspecifieke contexten zoals sonar, waar de optimale instellingen niet altijd intuïtief zijn. \\

FixMatch daarentegen stelt eenvoud centraal: het gebruikt een combinatie van pseudo-labeling op zwak geaugmenteerde voorbeelden, en eist vervolgens consistentie op sterk geaugmenteerde versies van dezelfde input – mits het model voldoende vertrouwen heeft in het pseudo-label. Deze aanpak maakt FixMatch efficiënt en relatief robuust, en bovendien minder afhankelijk van uitgebreid afgestelde parameters. De eenvoud van de methode betekent echter niet per se dat het altijd de beste prestaties levert, zeker niet wanneer het vertrouwen van het model in vroege training nog beperkt is.

\begin{table}[H]
    \centering
    \begin{tabular}{lll}
        \toprule
        \textbf{Eigenschap} & \textbf{FixMatch} & \textbf{MixMatch} \\
        \midrule
        Kernidee                            & \makecell[l]{Pseudo-labeling + \\ consistency} & \makecell[l]{consistency + \\ data mixing} \\
        Complexiteit                        & Eenvoudig                                    & Complexer \\
        Afhankelijkheid van tuning          & Laag                                         & Hoog \\
        Robuustheid                         & Hoog                                         & Matig \\
        Afhankelijkheid van augmentaties    & Ja (zwak + sterk)                            & Ja (mixing + sharpening) \\
        Prestaties op kleine datasets       & Goed                                         & Afhankelijk van tuning \\
        \bottomrule
    \end{tabular}
    \caption[Vergelijking semi-supervised modellen]{\label{tab:comparison_ssl_models} Tabel met een simpele vergelijking tussen de onderzochte \gls{ssl}-modellen binnen dit onderzoek.}
\end{table}

\subsection{Self-supervised learning}

\gls{byol}, \gls{moco} en \gls{simclr} vormen drie invloedrijke methoden binnen \gls{self-sl} die elk proberen om representaties te leren zonder gebruik te maken van gelabelde data. Alle drie maken gebruik van augmentaties en encoderstructuren om betekenisvolle visuele representaties te extraheren, maar verschillen fundamenteel in hoe zij het leerproces structureren -- vooral met betrekking tot het gebruik van contrastieve loss en negatieve voorbeelden. \\

\gls{simclr} is gebaseerd op contrastieve learning en vereist grote \glspl{batch_size} om voldoende negatieve paren te genereren, wat in praktijk computationeel intensief kan zijn. Het model leert door representaties van verschillende augmentaties van hetzelfde beeld dichter bij elkaar te brengen, en andere beelden juist uit elkaar te duwen. Deze aanpak is krachtig op grote, diverse datasets, maar kan kwetsbaar zijn in domeinen met beperkte data of lage variatie, zoals sonarbeelden. \\

\gls{moco} probeert de \gls{batch_size}-beperkingen van \gls{simclr} te omzeilen door gebruik te maken van een momentum-geüpdatete geheugenbank die een groot aantal negatieve representaties opslaat. Hierdoor is \gls{moco} efficiënter in resourcegebruik, maar het introduceert ook extra architecturale complexiteit en is gevoelig voor de consistentie van representaties over tijd -- iets wat uitdagend kan zijn bij sonardata, waar de semantische grenzen tussen objecten niet altijd scherp zijn. \\

\gls{byol} doorbreekt het contrastieve paradigma volledig door géén negatieve voorbeelden te gebruiken. In plaats daarvan traint het een online encoder om consistente representaties te leren ten opzichte van een langzaam geüpdatete target encoder. Deze aanpak is conceptueel eenvoudiger en robuuster, vooral in omgevingen met weinig visuele diversiteit of waar het definiëren van negatieve paren onnatuurlijk is. \gls{byol} vereist bovendien minder computationele resources en presteert stabieler bij kleinere datasets en \glspl{batch_size} -- eigenschappen die relevant zijn voor toepassing op sonardata, waar gelimiteerde data beschikbaar is en objecten visueel subtiel kunnen zijn.

\begin{table}[H]
    \centering
    \begin{tabular}{llll}
        \toprule
        \textbf{Eigenschap} & \textbf{\acrshort{simclr}} & \textbf{\acrshort{moco}} & \textbf{\acrshort{byol}} \\
        \midrule
        Gebruik van negatieve paren?        & Ja        & Ja                        & Nee \\
        \Gls{batch_size} vereisten          & Hoog      & Laag                      & Laag \\
        Architectuurcomplexiteit            & Gemiddeld & \makecell[l]{Hoog \\ (met geheugenbank)}   & Gematigd \\
        Stabiliteit bij kleine datasets     & Laag      & Gematigd                  & Hoog \\
        Prestaties bij weinig variatie      & Matig     & Matig                     & Goed \\
        \bottomrule
    \end{tabular}
    \caption[Vergelijking self-supervised modellen]{\label{tab:comparison_self_sl_models} Tabel met een simpele vergelijking tussen de onderzochte \gls{self-sl}-modellen binnen dit onderzoek.}
\end{table}


% Overzicht van bestaande datasets en annotatietechnieken
\input{standvanzaken/bestaande_datasets}
%%=============================================================================
%% Methodologie
%%=============================================================================

\chapter{Methodologie}%
\label{ch:methodologie}

Dit onderzoek volgt een gestructureerde aanpak om semi- en self-supervised learning technieken voor objectdetectie in sonardata te implementeren en te evalueren. In deze methodologie wordt een onderverdeling gemaakt van de verschillende fasen in dit onderzoek. Hierbij wordt de basis gelegd voor de experimenten die zullen worden uitgevoerd in de proof of concept.

\section{Data-acquisitie}

Allereerst moet er een keuze gemaakt worden voor het gebruik van een dataset in het verdere verloop van dit onderzoek. Alle drie de datasets die in \ref{subsec:mogelijke-oplossingen} besproken werden, maken het mogelijk om een objectdetectiemodel mee te trainen. Ze bieden namelijk allemaal annotaties van \glspl{bounding_box} op de bijhorende afbeeldingen aan. Daarnaast zijn deze datasets makkelijk te verwerken, aangezien alle beelden in conventionele afbeeldingsformaten zijn opgeslagen (zoals PNG, JPG, BMP, PGM, \dots). Toch is er -- specifiek voor dit onderzoek -- één dataset die geschikter is dan de anderen. De UATD-dataset is -- misschien ietwat subjectief -- uitgekozen om te gebruiken in de rest van dit onderzoek. Dit komt omdat ze bepaalde aspecten aanbiedt die de andere datasets niet hebben. \\

\gls{sss} for Mine Detection lijkt op het eerste zicht de perfecte dataset voor dit onderzoek. Het probleem is echter dat ze relatief klein is: ze bevat ``slechts'' 1170 afbeeldingen. Op het eerste zicht lijkt dit voldoende. Echter moet deze dataset nog opgesplitst worden in -- ten minste -- een trainingsset en een testset.\footnote{In een optimale situatie zou de data opgesplitst worden in drie sets: een trainingsset, een testset en een validatieset. Dit komt omdat de validatiedataset -- hoewel ze niet gebruikt wordt om het model te trainen -- gebruikt wordt om de paramaters van het model te tunen. Dit kan leiden tot \gls{overfitting}. Het is beter om als testset data te gebruiken dat het model nog niet gezien heeft. \autocite{Goodfellow_2016}} Ook komen er slechts 668 objecten voor in de dataset. Tot overmaat van ramp zijn deze ook zeer slecht verdeeld. 

\begin{table}[H]
    \centering
    \begin{tabular}{ll}
        \toprule
        \textbf{\# objecten / beeld} & \textbf{\# beelden} \\
        \midrule
        13 & 1 \\
        9  & 2 \\
        8  & 4 \\
        7  & 8 \\
        6  & 8 \\
        5  & 13 \\
        4  & 9 \\
        3  & 41 \\
        2  & 59 \\
        1  & 159 \\
        0  & 866 \\
        \bottomrule
    \end{tabular}
    \caption[Aantal objecten per afbeelding in SSS for Mine Data]{\label{tab:objects_per_image_sss} Tabel met verdeling van objecten per afbeelding in de \gls{sss} for Mine Detection-dataset.}
\end{table}

Er is één afbeelding met wel 13 objecten en 866 zonder ook maar één object. Door deze slechte verdeling en de beperkte hoeveelheid data in de dataset is ze dus weinig bruikbaar voor dit onderzoek. \\

Dit is een probleem waar de UXO-dataset absoluut niet mee kampt. Deze heeft dan echter weer andere problemen. De dataset is namelijk volledig samengesteld in een gecontroleerde testopstelling. Ze bevat dus geen \emph{real-world}-data. Dit betekent echter ook dat bepaalde artefacten en afwijkingen typisch aan meren en zeeën niet in deze dataset voorkomen. De beelden zijn zodanig zuiver dat het hoogstwaarschijnlijk mogelijk zou zijn om de \glspl{blindganger} te herkennen door te zoeken naar de groep helderste pixels of met een edge-detection algoritme. \autocite{Torre_1986} \\

Ook de grootte van de dataset is misschien iets te mooi om waar te zijn. De beelden in de dataset zijn namelijk geen onafhankelijke afbeeldingen, maar frames van een continue opname. Dit zorgt ervoor dat er (nagenoeg) geen verschil is tussen afbeelding $n$ en afbeelding $n+1$. Als alle afbeeldingen na elkaar worden afgespeeld, ziet men een opname van een transformatie (rotatie, verschuiving, \dots) van één van de \glspl{blindganger}. Ten slotte staat er telkens maar één object op een afbeelding, wat multiple objectdetectie (meerdere objecten op één afbeelding herkennen) onmogelijk maakt.

\section{Dataverdeling}

In dit onderzoek zullen verschillende modellen getraind worden met verschillende leertechnieken. Daarom is het belangrijk om een duidelijk zicht te krijgen op hoe de gekozen dataset verdeeld moet worden zodat dit efficiënt en effectief kan gebeuren. Zoals vermeld zal er gebruik gemaakt worden van de UATD-dataset. Aangezien deze al opgesplitst is in drie subsets, zal de data als volgt verdeeld worden:

\begin{itemize}
    \item \texttt{UATD\_Training}: trainingsset (7600 samples)
    \item \texttt{UATD\_Test\_1}: testset (800 samples)
    \item \texttt{UATD\_Test\_2}: validatieset (waar nodig/mogelijk) (800 samples)
\end{itemize}

De trainingsset bevat 7600 gelabelde samples. Om de hypotheses in dit onderzoek echter te kunnen testen, zal deze set om bepaalde modellen te trainen verder opgesplitst worden in een gelabelde en een ongelabelde set. Om de resultaten van \gls{ssl} en \gls{self-sl} beter te kunnen evalueren, zal en volledig gesuperviseerd model getraind worden op subsets van de gelabelde data. Meer specifiek gaat dit om: 100\% (7600 samples), 50\% (3800 samples), 10\% (760 samples), 5\% (380 samples) en 1\% (76 samples). In deze fase worden de overige ongelabelde samples gewoonweg niet gebruikt. \\

Omwille van de resource-intensiviteit om deze modellen te trainen en het feit dat dit ``slechts'' een proof of concept is, zullen de \gls{ssl}- en \gls{self-sl}-modellen dit trainingsschema niet volgen. Ze worden getraind op de twee subsets die het interessantst zijn voor dit onderzoek. Een subset van 10\% (760 samples) zal gebruikt worden om de praktische werking in een realistisch scenario aan te tonen. Daarnaast zal een subset van 5\% (380 samples) gebruikt worden om een mogelijk scenario waar data zeer schaars is na te bootsen. De overige samples zullen telkens gebruikt worden als ongelabelde trainingsdata. De modellen worden in se dus telkens op de volledige trainingsset getraind, echter is de leertechniek -- en dus de verhouding gelabeld/ongelabeld -- voor elk model verschillend.

\section{Modelselectie}

\subsection{Supervised: Faster R-CNN}



\subsection{Semi-supervised: FixMatch}

\subsection{Self-supervised: BYOL pre-training}

\section{Resultaten en evaluatie}

Uiteindelijk zullen de verschillende uitgevoerde experimenten worden geëvalueerd en met elkaar vergeleken. Dit zal op verschillende manieren gedaan worden. Specifiek gaat dit om kwantitatieve en kwalitatieve methoden. Kwantitatief zal vooral \acrfull{map} gebruikt worden om objectdetectieprestaties te beoordelen. Echter is dit niet de enige kwantitatieve metriek die er toe doet. Naast de effectieve performantie van het model is het ook belangrijk om rekening te houden met de resources die nodig zijn om een bepaald model te trainen. Ook wordt een vergelijking gemaakt tussen dezelfde modellen die getraind zijn met een verschillende verdeling van gelabelde en ongelabelde data. Dit om de label-efficiëntie (Hoe goed presteert het model met een beperkte hoeveelheid gelabelde data?) te meten. Dit zal cijfermatig uitsluitsel geven over de effectiviteit van semi-supervised en self-supervised learning tegenover supervised learning. \\

Naast de kwantitatieve analyse wordt er ook een kwalitatieve analyse uitgevoerd. De focus ligt hierbij niet zozeer op de cijfers, maar eerder op de effectieve voorspellingen van de modellen. Hierbij zal sterk worden gebruik gemaakt van beeldmateriaal en de voorspellingen gemaakt door de modellen. Hoewel een score heel objectief uitsluitsel kan geven over de performantie van een model, zijn real-world voorbeelden nog steeds waardevol voor de uiteindelijke evaluatie. Een score houdt namelijk niet altijd (genoeg) rekening met dingen die voor de eindgebruiker belangrijk zijn en omgekeerd. \\

Uiteindelijk zullen enkele experts in sonaranalyse de bruikbaarheid van de resultaten beoordelen en aanbevelingen geven voor verdere verbeteringen. Ten slotte zullen de methodologie, resultaten en code worden gedocumenteerd om reproduceerbaarheid te waarborgen.

% Corpus

\chapter{Data}
\label{ch:data}

Data is de backbone van dit onderzoek. Daarom is het belangrijk een betrouwbare en kwalitatieve dataset te gebruiken. Zoals eerder vermeld zal de UATD-dataset gebruikt worden. \\

%TODO
TODO: verder uitschrijven

\subsection{Herschalen van de afbeeldingen}

De meeste deep learning-modellen verwachten een input met consistente dimensies. 

\begin{listing}[H]
    \begin{minted}{python}
        def resize_image(image, target_size=(1024, 1024)):
            h, w = image.shape
            target_h, target_w = target_size
            
            # Calculate scaling factor
            scale = min(target_w / w, target_h / h)
            new_w = int(w * scale)
            new_h = int(h * scale)
            
            # Resize the image
            resized_image = cv2.resize(image, (new_w, new_h))
            
            # Create a new blank image with the target size
            new_image = np.zeros((target_h, target_w), dtype=np.uint8)
            
            # Calculate the padding
            pad_w = (target_w - new_w) // 2
            pad_h = (target_h - new_h) // 2
            
            # Place the resized image in the center of the new image
            new_image[pad_h:pad_h + new_h, pad_w:pad_w + new_w] = resized_image
            
            return new_image, (pad_w, pad_h, new_w, new_h)
    \end{minted}
    \caption[Functie \texttt{resize\_image}]{Functie die ervoor zorgt dat de afbeelding correct herschaald wordt en dat er \emph{padding} wordt toegevoegd om een afbeelding met consistente dimensies te bekomen.}
\end{listing}

% TODO: afbeelding toevoegen voor en na deze functie

De functie \texttt{resize\_image} past de grootte van een afbeelding aan en plaatst deze gecentreerd in een nieuw canvas met een vaste afmeting. De standaard \texttt{target\_size} is $1024 \times 1024$ pixels, maar dit kan worden aangepast. De functie gebruikt OpenCV (\texttt{cv2.resize}) om de afbeelding proportioneel te schalen, zodat deze in het doelcanvas past zonder de beeldverhouding te veranderen. \\

Eerst wordt de hoogte en breedte van de oorspronkelijke afbeelding bepaald, evenals de doelhoogte en -breedte. Vervolgens wordt een schaalfactor berekend op basis van de kleinste verhouding tussen de doelafmetingen en de oorspronkelijke afmetingen. De afbeelding wordt met deze schaalfactor aangepast, zodat deze zo groot mogelijk binnen de doelafmetingen past. \\

Daarna wordt een nieuw, zwart canvas aangemaakt met de doelgrootte. De berekende padding aan de randen zorgt ervoor dat de geschaalde afbeelding gecentreerd wordt geplaatst binnen dit canvas. Ten slotte retourneert de functie zowel de nieuw gecreëerde afbeelding als de padding-waarden, zodat men indien nodig de positie en grootte van de originele afbeelding in het nieuwe canvas kan achterhalen.

%TODO: beeldverhouding need gls

\subsection{Herberekenen van bounding boxes}

De functie \texttt{resize\_image} zorgt ervoor dat alle afbeeldingen herschaald worden naar consistente dimensies. Dit zorgt er echter voor dat de posities van de \glspl{bounding_box} niet meer kloppen. De volgende functie herberekent de \glspl{bounding_box}, rekening houdend met de herschaling en padding.

\begin{listing}[H]
    \begin{minted}{python}
        def convert_bounding_boxes(bboxes, padding, original_size, new_size=(1024, 1024)):
            pad_w, pad_h, new_core_w, new_core_h = padding
            init_img_w, init_img_h = original_size
            total_w, total_h = new_size
            
            normalized_bboxes = []
            
            for bbox in bboxes:
                x_min, y_min, x_max, y_max = bbox
                
                # Adjust for padding
                x_min = (x_min * new_core_w) / init_img_w + pad_w
                x_max = (x_max * new_core_w) / init_img_w + pad_w
                y_min = (y_min * new_core_h) / init_img_h + pad_h
                y_max = (y_max * new_core_h) / init_img_h + pad_h
                
                # Calculate the center and size of the bounding box
                bbox_width_in_px = x_max - x_min
                bbox_height_in_px = y_max - y_min
                x_center_in_px = x_min + bbox_width_in_px / 2
                y_center_in_px = y_min + bbox_height_in_px / 2
                
                # Normalize the coordinates
                x_norm = x_center_in_px / total_w
                y_norm = y_center_in_px / total_h
                bbox_width_norm = bbox_width_in_px / total_w
                bbox_height_norm = bbox_height_in_px / total_h
                
                normalized_bboxes.append((x_norm, y_norm, bbox_width_norm, bbox_height_norm))
            
            return normalized_bboxes
    \end{minted}
    \caption[Functie \texttt{convert\_bounding\_boxes}]{Functie die ervoor zorgt dat de \glspl{bounding_box} herberekend worden nadat de afbeelding herschaald is door \texttt{resize\_image}.}
\end{listing}

De functie \texttt{convert\_bounding\_boxes}, past de coördinaten van bestaande \glspl{bounding_box} aan nadat een afbeelding is geschaald en gecentreerd in een nieuw canvas. Dit is nodig omdat bij het aanpassen van de afbeelding de objectlocaties in de oorspronkelijke afbeelding niet langer direct overeenkomen met hun nieuwe positie in het vergrote of verkleinde en verschoven beeld. De \glspl{bounding_box} worden ook genormaliseerd, zodat ze geschikt zijn voor modellen die werken met schaalonafhankelijke coördinaten (bv. \gls{yolo}). \\

De functie ontvangt een lijst van \glspl{bounding_box}, de paddingwaarden van de afbeeldingstransformatie, de oorspronkelijke afbeeldingsgrootte en de nieuwe grootte (standaard $1024 \times 1024$ pixels). Voor elke \gls{bounding_box} worden de oorspronkelijke coördinaten aangepast aan de nieuwe afmetingen. Dit gebeurt door eerst de verhoudingen tussen de oude en nieuwe breedte en hoogte toe te passen en vervolgens de padding toe te voegen. Hierdoor worden de coördinaten correct in het nieuwe canvas geplaatst.

Daarna worden de aangepaste \gls{bounding_box}-coördinaten omgezet naar een gecentreerde representatie. Dit betekent dat de functie de breedte en hoogte van de box berekent en het middelpunt bepaalt. Vervolgens worden deze waarden genormaliseerd door ze te delen door de totale breedte en hoogte van het nieuwe beeld. Dit maakt de bounding box coördinaten onafhankelijk van de werkelijke afbeeldingsgrootte, waardoor ze compatibel zijn met (verschillende veelgebruikte) deep learning-modellen. De functie retourneert een lijst met genormaliseerde \glspl{bounding_box}, elk voorgesteld als \texttt{(x\_center, y\_center, width, height)}.
\chapter{Implementatie \& Optimalisatie}
\label{ch:implementatie}

De implementatie van de verschillende objectdetectiemodellen vormt de belangrijkste voorbereidende stap in het ontwikkelen van de ``proof of concept'' voor dit onderzoek. In deze fase wordt de theorie over de verschillende modelarchitecturen (zie \ref{ch:stand-van-zaken}) vertaald naar praktische, werkende systemen. Dit omvat het selecteren van een geschikte detectiemethode per categorie -- zowel supervised, semi-supervised en self-supervised -- en het configureren en connecteren van de verschillende componenten. Daarnaast wordt er aandacht besteed aan de optimalisatie-instellingen -- zoals learning rate schedules -- die van grote invloed (kunnen) zijn op de prestaties. Ook zal het gekozen deep learning-framework worden besproken en toegelicht. In dit hoofdstuk wordt beschreven hoe de gekozen modellen zijn geïmplementeerd, welke architecturale keuzes zijn gemaakt, en hoe deze implementatie is afgestemd op de specifieke kenmerken van de dataset en de uiteindelijke toepassing. Zo wordt de basis gelegd voor het daaropvolgende trainingsproces.

\section{Keuze van deep learning framework}

Vandaag de dag bestaan er verschillende deep learning frameworks die de ontwikkeling en implementatie van neurale netwerken sterk vergemakkelijken. Enkele van de meest gebruikte zijn TensorFlow, PyTorch, JAX en Keras. TensorFlow en PyTorch zijn de twee dominante deep learning frameworks in het veld van artificiële intelligentie. Keras is een speciale uitzondering: het is namelijk een bibliotheek die compatibiliteit biedt tussen deze verschillende frameworks.

\subsection{TensorFlow}

TensorFlow (geïntroduceerd door Google in 2015) werd ontworpen met het oog op grootschalige productieomgevingen. Het framework biedt uitgebreide ondersteuning voor het deployen van modellen op verschillende platformen, zoals mobiele apparaten en webapplicaties, en beschikt over geavanceerde tools zoals TensorBoard voor visualisatie en TensorFlow Serving voor schaalbare modelinzet. TensorFlow’s aanpak met statische computationele grafen (in vroege versies) maakte het echter aanvankelijk minder intuïtief voor onderzoek en experimentatie, hoewel TensorFlow 2.x dit deels heeft verbeterd door de introductie van \emph{eager execution}. TensorFlow biedt ook veruit de beste integratie met \glspl{tpu} (cf. \ref{subsec:tpu}).

\subsection{PyTorch}

PyTorch (ontwikkeld door \gls{fair} en uitgebracht in 2016) heeft daarentegen vanaf het begin sterk ingezet op gebruiksvriendelijkheid en flexibiliteit. PyTorch maakt gebruik van dynamische computationele grafen, wat betekent dat het netwerk direct kan worden aangepast tijdens de uitvoering. Dit maakt het debuggen eenvoudiger en laat onderzoekers sneller experimenteren met nieuwe architecturen en technieken. Bovendien sluit de programmeerstijl van PyTorch dichter aan bij standaard Python, wat de leercurve verlaagt en de ontwikkelsnelheid verhoogt. In de onderzoekswereld heeft PyTorch hierdoor snel populariteit gewonnen, en veel state-of-the-art modellen en papers publiceren tegenwoordig hun codebase standaard in PyTorch.

Voor dit project is uiteindelijk gekozen voor PyTorch vanwege de flexibiliteit en transparantie die het biedt tijdens het ontwikkelen en experimenteren met objectdetectiemodellen. PyTorch biedt namelijk uitstekende integraties met moderne detectiebibliotheken zoals TorchVision, wat de implementatie en evaluatie van complexe modellen versnelt. Daarnaast is het makkelijk om gepretrainde detectiemodellen te importeren, waardoor in dit onderzoek gefocust kan worden op de nieuwe ontwikkelingen zonder de focus te moeten verschuiven naar complexe backbones. Ook de installatie, configuratie en setup van PyTorch zijn veel eenvoudiger dan bij TensorFlow. Deze eigenschappen maken PyTorch tot de meest geschikte keuze om de doelstellingen van dit onderzoek efficiënt en effectief te realiseren.

\section{Implementatie van een gesuperviseerd model (Faster R-CNN)}


\chapter{Training \& Optimalisatie}
\label{ch:training-optimalisatie}

\lipsum[1-3]
\chapter{Evaluatie van de modellen}
\label{ch:evaluatie-modellen}

\lipsum[1-3]
\chapter{Evaluatie van de resultaten}
\label{ch:evaluatie-resultaten}

\lipsum[1-3]

%%=============================================================================
%% Conclusie
%%=============================================================================

\chapter{Conclusie}%
\label{ch:conclusie}

\section{Samenvatting van bevindingen}

Op basis van de uitgevoerde experimenten kan geconcludeerd worden dat zowel \gls{ssl} als \gls{self-sl} effectieve strategieën zijn om de afhankelijkheid van gelabelde data bij objectdetectie in sonarbeelden te reduceren. Het volledig gesuperviseerde baseline-model, Faster \gls{rcnn}, behaalde een hoge nauwkeurigheid wanneer het getraind werd op de volledige gelabelde dataset (100\%), met een \gls{map} van 0.7717. Bij afname van het gelabelde aandeel daalden de prestaties echter significant, met \gls{map}-scores van respectievelijk 0.7847 (50\%)\footnote{De hogere performantie van het 50\%-model ten opzichte van het 100\%-model kan verklaard worden door \gls{overfitting}: het model dat op 100\% van de data is getraind, heeft mogelijk meer ruis of irrelevante patronen in de volledige dataset geleerd. Bij 50\% gelabelde data is de kans groter dat het model generaliseert naar meer robuuste kenmerken, wat resulteert in een iets hogere \gls{map}.}, 0.6152 (10\%), 0.5298 (5\%) en 0.2799 (1\%). Deze trend onderstreept de kwetsbaarheid van conventionele supervised modellen in situaties met beperkte annotaties. \\

De toepassing van \gls{ssl} met FixMatch bood in dit opzicht een duidelijke verbetering. Wanneer slechts 5\% of 10\% van de data gelabeld was, behaalde FixMatch \gls{map}-scores van 0.6649 en 0.6828. Deze scores liggen beduidend hoger dan die van het supervised baseline-model bij dezelfde splits, wat aantoont dat het model in staat is om op effectieve wijze gebruik te maken van de resterende ongelabelde data via pseudo-labeling en consistency regularisatie. Hoewel FixMatch niet de absolute topresultaten behaalde, vormt het een krachtige techniek voor situaties waarin annotaties beperkt beschikbaar zijn, zonder dat pretraining noodzakelijk is. \\

De best presterende alternatieve aanpak bleek echter de \gls{self-sl} strategie met \gls{byol}. Door een \gls{byol}-model te pretrainen op ongelabelde sonarbeelden en deze gewichten vervolgens te gebruiken als backbone voor een Faster \gls{rcnn} model, konden bij 5\% en 10\% gelabelde data \gls{map}-scores van respectievelijk 0.6452 en 0.7230 bereikt worden. Met name de score van 0.7230 bij 10\% gelabelde data benadert de prestaties van het volledig gesuperviseerde model, ondanks dat het met slechts een tiende van de annotaties getraind werd. Dit suggereert dat de representaties geleerd tijdens de \gls{byol}-pretraining bijzonder effectief generaliseren naar de downstream detectietaak. \\

Samenvattend tonen de resultaten aan dat zowel FixMatch als \gls{byol} waardevolle technieken zijn in data-schaarsere omgevingen. FixMatch biedt een efficiënte manier om ongelabelde data direct tijdens training te benutten, terwijl \gls{byol} robuuste representaties leert die downstream training aanzienlijk versterken. De \gls{self-sl} aanpak met \gls{byol} leverde de beste algehele prestaties in low-label settings, en vormt daarmee een veelbelovende richting voor verdere optimalisatie van objectdetectie in gespecialiseerde domeinen zoals sonarbeeldanalyse.

\section{Reflectie op de methodologie}

Bij reflectie op de gehanteerde methodologie kan worden vastgesteld dat de keuzes die aan de basis van dit onderzoek lagen over het algemeen sterk onderbouwd en passend waren binnen de onderzoeksvraag. De geselecteerde dataset bleek bijzonder geschikt: ze bood voldoende variatie en detail om de relevantie van het probleem van objectdetectie in sonarbeelden realistisch te benaderen. Bovendien sloot de datastructuur goed aan bij de gekozen leerstrategieën, wat toeliet om zowel supervisie als pretraining correct te benutten. \\

Ook de selectie van modellen -- Faster \gls{rcnn}, FixMatch en \gls{byol} -- bleek achteraf gezien goed doordacht. Deze modellen zijn zowel conceptueel uitdagend als voldoende krachtig om degelijke experimentele resultaten op te leveren. Ze boden een goed evenwicht tussen theoretische diepgang en praktische toepasbaarheid. Tegelijkertijd moet worden erkend dat de implementatie van vooral de \gls{ssl} en \gls{self-sl} modellen niet zonder moeilijkheden verliep. Het FixMatch-model, hoewel goed gedocumenteerd in literatuur, bleek in praktijk lastig op punt te stellen: het verkrijgen van stabiele en competitieve prestaties vereiste veel iteraties, hyperparameter-tuning en finetuning van augmentatiestrategieën. Het model bleek gevoelig voor kleine variaties in input en training, wat het reproduceerbaar werken bemoeilijkte. \\

De moeilijkheden bij \gls{byol} waren nog fundamenteler in de vroege fase. Initieel bleek dat een fout in de modelconfiguratie ertoe leidde dat slechts een beperkt deel van de ResNet-architectuur effectief getraind werd. Deze fout beperkte de prestaties drastisch tot \gls{map}-scores onder 0.15, waardoor diepgaand debuggingwerk noodzakelijk was. Pas na het herstellen van deze fout kon het \gls{byol}-model zijn potentieel laten zien. Daarnaast was het uitvoeren van optimalisaties niet evident: kleine aanpassingen aan architectuur of trainingsstrategie hadden vaak geen effect of zelfs negatieve impact. Dit toont aan dat dergelijke modellen fragiel kunnen zijn en dat hun succes sterk afhankelijk is van zorgvuldige afstemming van alle onderdelen van de trainingspipeline. \\

Verder speelde de computationele requirements een belangrijke rol. Door de (relatief) lange trainingstijden en de nood aan krachtige \gls{gpu}-resources, moest het onderzoek in iteraties worden uitgevoerd en werd experimenteel werk vaak vertraagd. Sommige experimenten konden niet in parallel worden opgezet, wat tijdsdruk met zich meebracht. Deze beperkingen illustreren de praktische uitdagingen bij het toepassen van geavanceerde leermethodes in een onderzoeksomgeving die beperkt wordt qua resources. \\

Ondanks deze obstakels was de gekozen methodologie waardevol. De moeilijkheden brachten leerzame inzichten met zich mee over de robuustheid en gevoeligheid van moderne machine learning-technieken in domeinspecifieke contexten. Het experiment gaf een realistisch beeld van wat er komt kijken bij het inzetten van geavanceerde \gls{ssl} en \gls{self-sl} modellen in een complexe toepassing zoals sonarbeeldanalyse.

\section{Voorstellen voor verder onderzoek}

Hoewel dit onderzoek aantoont dat zowel \gls{ssl} als \gls{self-sl} grote voordelen kunnen bieden bij objectdetectie op sonarbeelden, blijven er nog verschillende onderzoekspistes open die verdere verkenning verdienen. Een eerste richting is het vergelijken van alternatieve \gls{self-sl} pretrainingstrategieën. Hoewel \gls{byol} in dit onderzoek werd gekozen vanwege zijn stabiliteit en prestaties zonder negatieve paren, zijn er andere competitieve methoden zoals \gls{simclr}, \gls{moco} en \gls{dino}. Een vergelijking van deze methodes zou kunnen uitwijzen welke het meest geschikt is voor de specifieke kenmerken van sonarbeeldvorming, zoals lage resolutie, ruis en beperkte kleurinformatie. Daarnaast kan onderzocht worden hoe verschillende backbone-architecturen (bijvoorbeeld ResNet, Swin Transformer of ConvNeXt) zich verhouden qua compatibiliteit en prestaties binnen deze pretrainingsstrategieën. \\

Ook op het vlak van \gls{ssl} zijn er waardevolle alternatieven die verder onderzoek nodig hebben. FixMatch toonde in dit onderzoek al sterke resultaten, maar recentere methoden zoals FlexMatch, SoftMatch of \gls{uda} hebben mogelijk nog betere prestaties, zeker in combinatie met meer geavanceerde of domeinspecifieke augmentatietechnieken. Denk hierbij aan augmentaties die beter aansluiten bij sonarcontexten, zoals het toevoegen van realistische ruis, reflectievariaties of vervormingen die voortkomen uit onderwaterpropagatie.

Een bijzonder interessante piste is de combinatie van \gls{self-sl} pretraining en \gls{ssl} fine-tuning. Door eerst representaties te leren via een \gls{self-sl} benadering zoals \gls{byol} en deze vervolgens verder te verfijnen met een methode als FixMatch, kunnen de voordelen van beide strategieën worden gecombineerd. Experimenteel onderzoek moet uitwijzen of deze hybride aanpak leidt tot hogere nauwkeurigheid en robuustere detectieprestaties in situaties met een minimale hoeveelheid gelabelde data. \\

Verder kan onderzoek naar transfer learning binnen het sonardomein waardevolle inzichten opleveren. Modellen die getraind zijn op grootschalige ruwe sonardata uit verschillende toepassingen -- bijvoorbeeld onderwaternavigatie, structuurinspectie of archeologisch onderzoek -- kunnen geëvalueerd worden op hun vermogen om te generaliseren naar nieuwe omgevingen of sensorconfiguraties. Dit zou toelaten om universele sonarbackbones te ontwikkelen die met minimale aanpassing bruikbaar zijn in diverse domeinen. \\

Ten slotte is er nood aan onderzoek dat zich richt op de interactie tussen modelprestatie en labelselectie. Actief leren, waarbij het model zelf aangeeft welke samples het meest informatief zijn om te labelen, biedt hier een kansrijke uitbreiding. Door actief leren te combineren met \gls{self-sl} of \gls{ssl} technieken, kan mogelijk een nog efficiëntere annotatiestrategie ontwikkeld worden. Daarnaast verdient ook foutenanalyse meer aandacht: een diepgaand inzicht in de soorten fouten die modellen maken, kan bijdragen aan interpretatie, vertrouwen en verdere optimalisatie van detectieprestaties in kritieke toepassingen zoals defensie of onderwaterrobotica.

%---------- Bijlagen -----------------------------------------------------------

\appendix

\chapter{Onderzoeksvoorstel}

Het onderwerp van deze bachelorproef is gebaseerd op een onderzoeksvoorstel dat vooraf werd beoordeeld door de promotor. Dat voorstel is opgenomen in deze bijlage.

\section*{Samenvatting}

Sinds de opkomst en popularisatie van AI-modellen is data steeds een cruciale resource geweest. Voor simpele modellen is de benodigde data vaak ook simpel van vorm en is er (relatief) weinig van nodig om een performant en goed werkend model te creëren. Echter stijgen de data-requirements voor grotere en complexere modellen exponentieel. De benodigde data om een objectdetectiemodel voor sonardata te trainen zorgt voor moeilijkheden: dit soort datasets zijn online niet off-the-shelf beschikbaar en zijn dus zeer tijdrovend en kostbaar om te maken. De hoofdvraag van dit onderzoek is daarom: Op welke manieren kan het gebruik van semi- of self-supervised learning het labelproces versnellen zonder significante verlies in nauwkeurigheid? Door verschillende technieken toe te passen, zal een pretraining-strategie ontwikkeld worden die gebruikmaakt van ongelabelde data om representaties aan te leren. Vervolgens zal onderzocht worden hoeveel gelabelde data nodig is om een goed presterend detectiemodel te trainen. Het doel is een methodologie te ontwikkelen die de afhankelijkheid van handmatig gelabelde data minimaliseert, terwijl de prestaties van het detectiemodel behouden blijven. De resultaten kunnen bijdragen aan efficiëntere workflows voor data-analyse in sonarbeeldvorming en andere domeinspecifieke contexten.

\newpage

% Verwijzing naar het bestand met de inhoud van het onderzoeksvoorstel
%---------- Inleiding ---------------------------------------------------------

\section{Inleiding}%
\label{sec:inleiding}

Objectdetectie in sonardata speelt een cruciale rol in toepassingen zoals onderwaterverkenning, maritieme veiligheid en archeologisch onderzoek. Traditionele methoden voor objectdetectie vereisen echter grote hoeveelheden gelabelde data, wat een tijdrovend en duur proces is. Sonarafbeeldingen, gekenmerkt door ruis en unieke reflectiepatronen, vergen bovendien gespecialiseerde kennis voor annotatie. Dit maakt het labelen van datasets een grote uitdaging, vooral wanneer schaal en diversiteit van de data toenemen.

Semi- en self-supervised learning bieden veelbelovende alternatieven om dit probleem te overbruggen. Deze technieken maken gebruik van ongesuperviseerde data om representaties te leren en beperken de afhankelijkheid van gelabelde gegevens. Moderne self-supervised methoden hebben indrukwekkende resultaten laten zien in domeinen zoals computer vision, maar hun toepassing op domeinspecifieke datasets, zoals sonar, is nog relatief onbekend terrein.

Dit onderzoek richt zich op de vraag of semi- en self-supervised learning technieken effectief kunnen worden ingezet om het labelproces in sonarobjectdetectie te versnellen, zonder dat dit ten koste gaat van de nauwkeurigheid van het model. Door technieken aan te passen aan de specifieke eigenschappen van sonardata, wordt onderzocht hoe representaties kunnen worden geleerd die bijdragen aan een efficiëntere en kosteneffectieve workflow.

Naast de vergelijking tussen semi-supervised / self-supervised en supervised learning, zullen verschillende details onderzocht worden. Dit gaat van hoeveelheid gelabelde data minimaal nodig om met semi-supervised learning vergelijkbare resultaten te behalen als volledig gesuperviseerde methoden tot pre-training-methoden uit self-supervised learning die het meest geschikt zijn voor sonardata. Daarnaast wordt onderzocht welke aanpassingen nodig zijn om bestaande semi- en self-supervised technieken effectief toe te passen op stacked GeoTIFF-sonardata. Uiteindelijk wordt uitgezocht welke evaluatiemethoden het beste de prestaties van semi- en self-supervised modellen in sonarobjectdetectie kunnen meten.

%---------- Stand van zaken ---------------------------------------------------

\section{Literatuurstudie}%
\label{sec:literatuurstudie}

Objectdetectie in domeinspecifieke contexten zoals sonarbeeldvorming wordt vaak gehinderd door een gebrek aan gelabelde data. Traditioneel vereisen gesuperviseerde modellen grote hoeveelheden handmatig gelabelde gegevens om effectieve detectie en classificatie te leren. Semi-supervised en self-supervised learning bieden echter veelbelovende alternatieven door gebruik te maken van grote hoeveelheden ongesuperviseerde data om representaties te leren. Deze literatuurstudie bespreekt de huidige technieken en hun toepassing, met een specifieke focus op de unieke uitdagingen van sonardata.

\paragraph{Gesuperviseerde Objectdetectie}

Objectdetectie is een tak binnen het domein van computer vision dat gericht is op het identificeren en lokaliseren van objecten binnen beelddata (zoals foto's en video's). Dit wordt gebruikt in verschillende domeinen, zoals beveiligingssystemen (bv. om inbrekers te detecteren) of de medische wereld (bv. om tumoren op te sporen). Door de jaren heen is objectdetectie aanzienlijk geëvolueerd dankzij de vooruitgang in Deep Learning en de grote beschikbaarheid van datasets met beeldmateriaal. \autocite{He_2016} 

Objectdetectie combineert twee belangrijke zaken in computer vision: objectlokalisatie en objectclassificatie. Objectlokalisatie bepaalt de positie van objecten, meestal in de vorm van bounding boxes \autocite{Tompson_2015}, terwijl objectclassificatie bepaalt tot welke categorie een gedetecteerd object behoort. Samen geeft dit de mogelijkheid tot het herkennen van verschillende objecten op één foto.

Traditioneel worden gesuperviseerde methoden -- zoals Faster R-CNN, YOLO en SSD -- gebruikt voor objectdetectie. \autocite{Redmon_2016} Deze modellen presteren uitstekend bij voldoende gelabelde data, maar de annotatiekosten en tijdsinvestering vormen een grote belemmering, vooral bij complexe datasets zoals sonar. Sonardata vereist namelijk gespecialiseerde kennis voor het labelen, wat de annotatie nog uitdagender maakt. \autocite{Long_2015}

\paragraph{Semi-supervised learning}

Binnen het domein van machine learning bestaan er verschillende technieken om een model te trainen. Meestal wordt er gesproken van twee grote stromingen: supervised learning en unsupervised learning. Bij supervised learning wordt er gebruik gemaakt van een dataset en een uitkomst (hetgeen het model uiteindelijk moet kunnen voorspellen). Dit kan een label zijn of een bepaalde numerieke waarde. Belangrijk is dat zowel de input als de gewenste output gegeven zijn. Het model leert dus het verband tussen de twee. Bij unsupervised learning zijn er geen verwachte outputs. De volledige dataset wordt door het model gebruikt om patronen in te herkennen. Unsupervised learning wordt daarom ook meestal gebruikt om verkennende data-analyse uit te voeren. Echter hebben beide methoden enkele nadelen. Bij supervised learning is het traag en duur om alle data op een correcte manier te labelen. Unsupervised learning heeft dit probleem niet, maar heeft een beperkt aantal toepassingen en is minder accuraat. Een alternatief is semi-supervised learning, wat een compromis tussen zowel supervised als unsupervised learning is. \autocite{C_A_Padmanabha_Reddy_2018}

Semi-supervised learning maakt gebruik van zowel gelabelde als ongelabelde data. Het algoritme traint eerst een basismodel op de beperkte hoeveelheid gelabelde data, waarna dit gebruikt wordt om de grotere ongelabelde dataset van labels te voorzien. Deze techniek wordt ook wel Pseudo-labeling genoemd. \autocite{Lee_2013}

Een andere populaire techniek binnen semi-supervised learning is Consistency Regularization. Deze techniek zorgt ervoor dat het model consistente voorspellingen aanleert door gebruik te maken van data-augmentatie. De inputdata wordt hierbij licht aangepast of verstoord. Het uitgangspunt is dat een goed model robuust moet zijn tegen kleine veranderingen in de input. \autocite{Fan_2022}.
Deze twee populaire technieken kunnen ook gebruikt worden om een nog beter model te maken zoals in het FixMatch-algoritme van \textcite{Sohn_2020}.

\paragraph{Self-supervised learning}

Anders dan bij semi-supervised learning is er bij self-supervised learning of SSL helemaal geen nood aan labels. Deze creëert het algoritme namelijk zelf tijdens een pre-training fase. Semi-supervised learning behoort echter niet helemaal tot het domein van unsupervised learning, hoewel het gebruik maakt van verschillende groeperings- en clusteringmethoden. Het uiteindelijke model maakt namelijk gebruik van -- door de pre-training -- gelabelde data. Deze techniek zorgt ervoor dat er veel complexere modellen getraind kunnen worden zonder een gigantische hoeveelheid aan data. Enkele nadelen zijn wel dat de techniek een grote hoeveelheid computerkracht nodig heeft en een lagere accuratie heeft dan supervised learning. \autocite{Gui_2024}

Ondanks deze nadelen blijken verschillende populaire technieken succesvol binnen het domein van computer vision.

\begin{itemize}
    \item SimCLR: deze afkorting staat voor Simple Framework for Contrastive Learning of Visual Representations. Deze techniek leert door gelijkenissen en verschillen tussen gegevensvoorbeelden te vergelijken, gebruikmakend van augmentaties zoals rotatie en kleurverandering. Het model wordt getraind om representaties van dezelfde input dichter bij elkaar te brengen (positieve paren) en die van verschillende inputs verder uit elkaar te houden (negatieve paren). \autocite{Chen_2020}
    \item BYOL: deze afkorting staat voor Bootstrap Your Own Latent. Deze techniek gebruikt twee netwerken: een online netwerk en een target netwerk. Het online netwerk leert een representatie van de ene augmentatie. Hierna genereert het target netwerk een representatie van de andere augmentatie. Uiteindelijk wordt het model getraind om de representatie van het online netwerk zo dicht mogelijk te laten lijken op die van het target netwerk. Merk op dat BYOL geen negatieve paren nodig heeft, wat leidt tot eenvoudigere training en vaak betere resultaten. \autocite{Grill_2020}
\end{itemize}

Zowel SimCLR als BYOL hebben voordelen en nadelen. Toch zijn ze allebei uitermate geschikt voor de use-case in dit onderzoek. Hieronder worden enkele verschillen besproken:

\begin{itemize}
    \item Contrastive learning: SimCLR gebruikt contrastieve loss met zowel positieve als negatieve paren, 
    terwijl BYOL gebruikt enkel positieve paren gebruikt.
    \item Efficiëntie: BYOL heeft minder geheugen en computationele middelen nodig omdat er geen grote batch negatieve voorbeelden nodig zijn.
    \item Complexiteit: SimCLR is eenvoudiger qua architectuur terwijl BYOL een dubbele netwerkstructuur nodig heeft.
\end{itemize}

Beide methoden hebben bewezen krachtig te zijn in self-supervised learning en kunnen worden aangepast voor domeinspecifieke data, zoals sonarafbeeldingen.

\paragraph{Optimalisatie van sonardata}

Sonardata vormt een unieke uitdaging voor objectdetectie vanwege de specifieke kenmerken van akoestische beelden, zoals ruis, lage resolutie en reflecties. De meeste technieken zijn echter oorspronkelijk ontwikkeld voor visuele data. Er zijn dus verschillende aanpassingen nodig om deze methoden effectief te laten werken op sonarafbeeldingen. Één van deze aanpassingen zit hem in het pre-processen van de data met behulp van filters om ruis te onderdrukken en het contrast te verbeteren. Ook kan er gebruik gemaakt worden van aangepaste modellen die speciaal kunnen worden ontworpen om beter te werken met sonardata. Dit kan door rekening te houden met hoe objecten in de data ruimtelijk met elkaar verbonden zijn en met specifieke geluidspatronen die in sonarbeelden voorkomen. \autocite{Karimanzira_2020}

\paragraph{Formaat van de data}

Daarnaast is het formaat van de data cruciaal. Er wordt (hoogstwaarschijnlijk) gebruik gemaakt van stacked GeoTIFF. Hierbij wordt het GeoTIFF formaat gebruikt om meerdere lagen geografische data te combineren in één bestand. GeoTIFF is een uitbreiding van het standaard TIFF (Tagged Image File Format) en wordt gebruikt om rasterafbeeldingen te koppelen aan geografische coördinaten. \autocite{Ritter_1997}

\paragraph{Uitdagingen \& opportuniteiten}

De vorige paragrafen bespraken vooral het technische aspect van semi-supervised en self-supervised learning. De algemene consensus is dat deze technieken veelbelovend zijn om, bijvoorbeeld, een computervisie model te trainen. Een ander domein waar deze technieken zeer goed van pas komen, is NLP. Toch bestaan er enkele uitdagingen, al dan niet voor deze specifieke use-case:

\begin{itemize}
    \item Modeloptimalisatie: sonarafbeeldingen vereisen aangepaste augmentaties en loss-functies.
    \item Dataschaal en kwaliteit: ongesuperviseerde sonardata kan ruis bevatten die het leerproces beïnvloedt.
    \item Prestatievergelijking: het vaststellen van de minimale hoeveelheid gelabelde data die nodig is om vergelijkbare resultaten te bereiken als volledig gesuperviseerde methoden.
\end{itemize}

\paragraph{Conclusie}

De literatuur toont aan dat semi- en self-supervised learning krachtige technieken zijn om de afhankelijkheid van handmatige labeling te verminderen. Door deze methoden aan te passen aan de unieke eigenschappen van sonardata, zoals ruis en textuurkenmerken, kan de efficiëntie van het labelproces aanzienlijk worden verbeterd. Dit biedt een veelbelovende richting voor verder onderzoek en praktische toepassingen in sonarbeeldanalyse.

%---------- Methodologie ------------------------------------------------------
\section{Methodologie}%
\label{sec:methodologie}

Dit onderzoek volgt een gestructureerde aanpak om semi- en self-supervised learning technieken te evalueren voor objectdetectie in sonardata. De methodologie is onderverdeeld in zes fasen: literatuurstudie, data pre-processing, modelontwikkeling, training en evaluatie.

\paragraph{Fase 1}

De eerste fase richt zich op het verzamelen van gedetailleerde informatie over het domein en de huidige stand van zaken hierbinnen. Specifiek gaat dit om het verzamelen, bestuderen en analyseren van wetenschappelijke artikelen. Op basis hiervan zal het onderzoek verder uitgewerkt worden.

\paragraph{Fase 2}

De tweede fase richt zich op het verzamelen, pre-processen en verdelen van de benodigde data voor dit onderzoek. Er wordt gebruik gemaakt van sonarafbeeldingen in gestapeld GeoTIFF-formaat. Deze data bevat verschillende lagen, zoals intensiteit en diepte-informatie, die dienen als input voor het model. Daarna zal pre-processing op de data toegepast worden. Er zal gebruik gemaakt worden van normalisatie, waarbij ruis en artefacten worden verminderd door technieken zoals median filtering. Daarnaast zullen augmentaties zoals rotatie, ruisinjectie en schaling worden toegepast om variatie in de dataset te vergroten en robuustheid van het model te verbeteren. Uiteindelijk wordt de data opgedeeld in een gelabelde en een grotere ongesuperviseerde subset. Dit maakt het mogelijk om zowel semi- als self-supervised technieken te testen.

\paragraph{Fase 3}

De derde fase richt zich op de ontwikkeling van de modellen zelf. Een baseline-model zoals een convolutioneel neuraal netwerk (bijvoorbeeld Faster R-CNN of YOLO) wordt gebruikt als referentiepunt voor volledig gesuperviseerde prestaties. Daarna worden self-supervised- of semi-supervised-modellen getraind worden ter vergelijking met het baseline-model. Binnen semi-supervised learning zal er geëxperimenteerd worden met pseudo-labeling: ongesuperviseerde voorbeelden worden automatisch gelabeld door het model en toegevoegd aan de trainingsset. Ook technieken zoals FixMatch om consistente voorspellingen te leren zullen bekeken en besproken worden. Binnen self-supervised learning zullen methoden zoals SimCLR of BYOL toegepast worden om representaties te leren zonder labels. Ook zullen aanpassingen aan pretext-taken, zoals het voorspellen van ontbrekende delen van sonarafbeeldingen of contrastieve augmentaties die rekening houden met spatiële afhankelijkheden besproken worden.

\paragraph{Fase 4}

De vierde fase richt zich op het trainen en optimaliseren van de ontwikkelde modellen. Dit omvat de pre-training, waarbij het model getraind wordt met ongesuperviseerde data om algemene representaties te leren. Daarnaast zal het model fine-tuning ondergaan: na pretraining wordt het model verder getraind met een kleine gelabelde dataset om objectdetectie te verfijnen. Ook hyperparameter-tuning behoort tot deze fase. Hierbij worden parameters zoals leersnelheid, batchgrootte en augmentatie-instellingen geoptimaliseerd om de prestaties te verbeteren.

\paragraph{Fase 5}

De vijfde fase richt zich op de evaluatie van de verschillende modellen. Er worden verschillende metrieken gebruikt om de modellen te evalueren. Onder andere Mean Average Precision (mAP) voor objectdetectieprestaties, Intersection over Union (IoU) voor lokalisatienauwkeurigheid en label-efficiëntie (Hoe goed presteert het model met een beperkte hoeveelheid gelabelde data?) zullen gebruikt worden. Er zal ook een vergelijking gemaakt worden met het baseline-model. Dit zal uitsluitsel geven over de effectiviteit van semi-supervised en self-supervised learning tegenover supervised learning. Uiteindelijk zullen enkele robuustheidstests uitgevoerd worden. Het model wordt hierbij getest op ongeziene data met variërende omstandigheden, zoals ruisniveaus en objecttypes.

\paragraph{Fase 6}

De zesde fase richt zich op de validatie en praktische toepassing van het model. Hierbij worden de modellen getest op een onafhankelijke dataset van sonarafbeeldingen om generaliseerbaarheid te evalueren. Ook zullen enkele experts in sonaranalyse de bruikbaarheid van de resultaten beoordelen en aanbevelingen geven voor verdere verbeteringen. Ten slotte zullen de methodologie, resultaten en code worden gedocumenteerd om reproduceerbaarheid te waarborgen.

%---------- Verwachte resultaten ----------------------------------------------
\section{Verwacht resultaat, conclusie}%
\label{sec:verwachte_resultaten}

Het onderzoek verwacht aan te tonen dat semi- en self-supervised learning effectief kunnen worden ingezet om het labelproces bij sonarobjectdetectie aanzienlijk te verminderen. Door technieken zoals SimCLR of BYOL te gebruiken voor pre-training, wordt verwacht dat het model sterke representaties leert van unsupervised sonardata, wat de behoefte aan grootschalige gelabelde datasets verkleint. Daarnaast zal een analyse inzicht geven in de minimale hoeveelheid gelabelde data die nodig is om vergelijkbare of betere prestaties te behalen dan met volledig supervised-learning methoden. Dit resulteert in een efficiëntere en kosteneffectieve aanpak voor objectdetectie in sonarbeelden, zonder verlies van nauwkeurigheid, en biedt een waardevolle methodologie voor verdere toepassingen in domeinen waar gelabelde data schaars is.



%%---------- Andere bijlagen --------------------------------------------------

% . . .

%%---------- Backmatter, referentielijst ---------------------------------------

\backmatter{}

\setlength\bibitemsep{2pt} %% Add Some space between the bibliograpy entries
\printbibliography[heading=bibintoc]

\end{document}
